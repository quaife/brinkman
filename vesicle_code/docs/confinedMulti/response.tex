\documentclass[11pt]{article}

\usepackage{fullpage}
\usepackage{url}

\begin{document}

We thank the referees for their assessment of our work.  The following
is our response to each of the specific comments and suggestions.

%%%%%%%%%%%%%%%%%%%%%%%%%%%%%%%%%%%%%%%%%%%%%%%%%%%%%%%%%%%%%%%%%%%%%%%%
\subsection*{Reviewer 1}

The referee wrote: I have reviewed ''Hydrodynamics of Multicomponent
Vesicles Under Strong Confinement'' by Gannon, Quaife, and Young. To my
understanding this work is unique as the dynamics of multicomponent
vesicles under constrictions has not yet been investigated. Before
acceptance there are some questions and issues which should be
addressed:
 
\noindent
{\bf Response:}: We thank the referee for the confirmation that the
topic is unique and has not yet been investigated.

\begin{enumerate}
%%%%%%%%
\item In the phase energy, Eq. (5), it is stated that the parameter
  $\epsilon$ is much smaller than 1 but then it is stated that this
    parameter is equal to 100. Please address this inconsistency.

\noindent
{\bf Response:} This was our oversight. The parameter values for $a$ and
$\epsilon$ were unintentionally reversed. This has been fixed.

%%%%%%%%
\item What motivated the choice of a sigmoid function, Eq. (9), rather
  than a simple min/max operation so that the bending rigidity does not
    extend past the $[0.1, 1]$ range? Would the use of a regular-mixture
    model, which might be more physically realistic, remove the need for
    such a mathematical approximation as the equilibrium concentrations
    will no longer be $0$ and $1$?

\noindent
{\bf Response:} A response

%%%%%%%%
\item Please define $\eta$ in Eq.~(10). 

\noindent
{\bf Response:} $\eta$ is a density function that is required to satisfy
    the Dirichlet boundary condition on the solid walls. The text has
    been adjusted.

%%%%%%%%
\item Please define whether $\mathbf{n}$ in Eq.~(12) is the inward or outward pointing unit normal.

\noindent
{\bf Response:} It is the outward pointing unit normal. The text has
    been adjusted.

%%%%%%%%
\item How is surface incompressibility enforced?

\noindent
{\bf Response:} The tension $\sigma$ acts as a Lagrange multiplier so
    that the inextensibility condition can be satisfied. The text has
    been adjusted.

%%%%%%%%
\item Please define $\mathbf{U}$ in Eq. (15). Is this the velocity of
  the interface?

{\bf Response:} This is the prescribed velocity along the solid wall
    which is a combination of no-slip along the top and bottom of the
    channel, and a Poiseuille flow at the inlet and outlet. We now
    reference equation (3), which defines $\mathbf{U}$, immediately
    after equation (15). 

%%%%%%%%
\item What value of $\delta$ is used to determine the lubrication layer
  thickness?

{\bf Response:} A response

%%%%%%%%
\item The simulation parameters must be provided. In the stenosis case
  what is the (dimensionless) width of the constricted region? For the
    contracting geometry case what does a $2~\mu$m wide neck correspond
    to?

{\bf Response:} A response

\end{enumerate}

\newpage 
%%%%%%%%%%%%%%%%%%%%%%%%%%%%%%%%%%%%%%%%%%%%%%%%%%%%%%%%%%%%%%%%%%%%%%%%
\subsection*{Reviewer 2}

The referee wrote: The authors of the paper, explored the effect of
hydrodynamics on the behavior of multicomponent vesicles under strong
confinement using numerical simulations. They have elucidated the
effects of the lubrication layer that exists between the vesicle, and
explored how confinement affects the excess pressure needed to push the
vesicle through the conduit. This study is useful to the community and
contains some sound insights regarding multicomponent vesicles. It,
however, is important that the authors make the following changes to be
considered for publication.

\noindent
{\bf Response:}: We thank the referee for their assessment and suggested
changes.

\begin{enumerate}
%%%%%%%%
\item The authors specify a line energy parameter $\epsilon = 100$. It
  is important that the authors justify this choice along with an
    experimental study which uses similar parameter ranges. This is
    important so that a future experimental study could be performed
    using these parameters as a guide.

\noindent
{\bf Response:} This was our oversight. The parameter values for $a$ and
$\epsilon$ were unintentionally reversed. This has been fixed.

%%%%%%%%
\item Minor comment: In Figure 4 and 5, the authors should add the $x$
  coordinate under the figure so that the readers are able to understand
    the results better.

\noindent
{\bf Response:} A response

%%%%%%%%
\item The authors mention the phrase `size of lubrication layer' in the
  paper. They need to explicitly define this parameter. It could be
    confusing to a reader whether they talk about
    $|\gamma_{\mathrm{layer}}|$ or $w_{\mathrm{top}}$ and
    $w_{\mathrm{bottom}}$.

\noindent
{\bf Response:} A response

%%%%%%%%
\item The authors measure the tangential velocity at 4 points. They
  should provide some justification for this number. Also, it is
    advisable that in Figure 7, the authors use a different color for
    each velocity plot. The plot looks highly convoluted and does not
    convey the message clearly. Also, what are the bounds within which
    the authors call the tangential velocity to be constant in Figure 7?
    They should define a bound for the variation that could be
    considered constant. Moreover, in Figure 7, for the floppy vesicle,
    the authors see a large deviation for the tangential velocity. It
    doesn't seem like the velocity could be called `constant'. The
    authors should provide a more detailed explanation for this.

\noindent
{\bf Response:} A response

%%%%%%%%
\item In some simulations, the authors choose a `random' initial
  condition. It would be helpful if they could mathematically define the
    function for this initial condition.

\noindent {\bf Response:} The examples with a random initial condition
    use the uniform distribution with the range $[0,1]$ to define the
    concentration $u$ at each spatial point. Given the choices for
    $b_{\min}=0.1$ and $b_{\max}=1$, the resulting mean bending
    stiffness is $\overline{b(u)} = 0.55$. The text has been adjusted at
    the start of Section 3.1 and Section 3.2.

%%%%%%%%
\item In Figure 6, the author should comment on the energy behavior of
  the floppy vesicle in the stenosis region.

\noindent
{\bf Response:} A response

%%%%%%%%
\item In Figure 7, when the authors look at the excess pressure plots,
  they should describe the behavior of the floppy vesicle due to the
  difference in lubrication thickness and, in turn, excess pressure. It
  seems like the excess pressure shows a similar trend as the stiff
  vesicle.

\noindent
{\bf Response:} A response


%%%%%%%%
\item Not really a comment, but the author's analysis of the velocity
  profiles in Figure 8 is fantastic. It captures the stress and pressure
  gradient behavior aptly and provides a theoretical basis as well.

\noindent
{\bf Response:} We thank the reviewer for this compliment. We are also
very happy with this result. 


%%%%%%%%
\end{enumerate}

\newpage

%%%%%%%%%%%%%%%%%%%%%%%%%%%%%%%%%%%%%%%%%%%%%%%%%%%%%%%%%%%%%%%%%%%%%%%%
\subsection*{Reviewer 3}

The referee wrote: The authors study the passage of a vesicle through a
constriction. A vesicle’s membrane has some mechanical properties analog
to red blood cells motivating many theoretical and numerical studies.
Indeed, this issue has known many progress during the last decade. A
vesicle is also a fascinating physical soft particle as its membrane
area is constant as well as its volume leading to an essential role of
deflation. But contrary to capsules, as the membrane is fluid, there is
no buckling. Finally, the study is inspired by modeling RBC. 

\noindent
{\bf Response:}: We thank the referee for their assessment and suggested
changes.

\begin{enumerate}
%%%%%%%%
\item This study is limited to the two-dimensional case, which is not
  reality. It is a strong approximation and full 3D studies are still a
    challenge in constrictions. But, in the axisymmetrical case, there
    are several results in literature which could be more or less,
    considered as benchmark:
    \url{http://dx.doi.org/10.1103/PhysRevFluids.5.043602}, \\
    \url{https://doi.org/10.1017/jfm.2017.743}

\noindent
{\bf Response:} A response

%%%%%%%%
\item The authors should justify why they limit their investigations to
  the 2D case if 3D is possible. I note that the authors also cite
    relevant literature as 10, 11 $\ldots$ References to 2D are
    excellent. A typical argument is the high time of 3D simulations.
    Can the authors provide an order of magnitude?

\noindent
{\bf Response:} A response

%%%%%%%%
\item The authors present the essential steps of the numerical model. It
  is well done and the references well chosen. I have appreciated the
    introduction of equation (9) an figure 2. However, there are two
    points to clarify :
    \begin{itemize}
      \item page 2 : there is a difference in literature of the
        capillary number. Some authors use the flow curvature, others
        the mean velocity as the authors. Please explain your choice.
      \item page 2, right column ``The parameter $\epsilon \ll 1$ sets
        the size of the transition region of u, and the parameter a is
        line tension scaled by the characteristic bending stiffness. All
        simulations use the parameter values $\epsilon = 100$ and $a =
        0.04.$" $\epsilon = 100$ is contradictory to $\epsilon \ll 1$.
        Please explain or correct
    \end{itemize}

\noindent
{\bf Response:} This was our oversight. The parameter values for $a$ and
$\epsilon$ were unintentionally reversed. This has been fixed.

%%%%%%%%
\item Why do authors never observe the blocking of vesicles contrary to
  experiments? It would the opportunity to perform a phase diagram and
    to show the effect of multicomponent.

\noindent
{\bf Response:} A response

%%%%%%%%
\item In experiments, the channel is a slit. What do the authors expect
  from such a difference?

\noindent
{\bf Response:} A response

%%%%%%%%
\item Finally, this study is interesting and provides new insights.
  However, I advise authors to improve the quantitative part of their
  results (variation with Ca or confinement or something else, phase
  diagram...) to be more relevant for a journal such as Soft Matter. I
  am waiting the new manuscript.

\noindent
{\bf Response:} A response

%%%%%%%%
\item Page 2, left column : the flow is assumed to be in the limit of
  zero Reynolds number : not the fluid

\noindent
{\bf Response:} A response

%%%%%%%%
\item The title is a bit misleading as the authors only consider the
  deflation equal to the one of RBC. However, as there is some conflict
  between capsules and vesicles as models of RBC, the title can be
  tolerated.

\noindent
{\bf Response:} A response
\end{enumerate}


\end{document}
