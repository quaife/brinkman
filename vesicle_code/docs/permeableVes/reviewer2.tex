\documentclass[11pt]{article}

\usepackage{fullpage}
\usepackage{color}
\newcommand{\response}[1]{\textcolor{blue}{#1}}

\begin{document}
We thank the reviewer for reviewing our manuscript \textit{Hydrodynamics
of a semipermeable inextensible membrane under flow and confinement}.
Below is a point-by-point response addressing each of the referee's
comments.
\begin{enumerate}
  \item I am slightly confused as to how the length and time scales are
    derived from the parameters discussed on page 4. As the simulation
    is 2D, it would seem to me that all physical parameters can only be
    defined on a ``per unit of length" basis. E.g., for a 2D vesicle
    contour, the surface tension is actually a line tension with units
    of N, not N/m. Could the authors clarify this?  In particular,
    reference is often made to how long things will take in the results
    (seconds, minutes, days, weeks, etc.), which is why I ask how these
    time scales are obtained. \\ \\
    \response{A response}

  \item Could the authors shed some insight as to how these 2D
    simulations could be reliably connected to real experiments on any
    quantitative basis? Or is the connection purely qualitative? \\ \\
    \response{A response}

  \item On the last page: ``We observe lose of" $\rightarrow$ ``We
    observe loss of". \\ \\
    \response{Thank you for noticing this typo. It is corrected.}


\end{enumerate}


\end{document}
