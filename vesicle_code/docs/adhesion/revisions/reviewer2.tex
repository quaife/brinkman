\documentclass[11pt]{article}

\usepackage{fullpage}
\usepackage{todonotes}
\usepackage{amsmath,amsfonts,amssymb,stmaryrd}
\usepackage{color}
\newcommand{\comment}[1]{{\color{blue} #1}}
\newcommand{\xxi}{\boldsymbol{\xi}}

\begin{document}
\noindent
Thank you for your constructive comments.  The attached manuscript
addresses all the reviewer 2's comments and the result is what we
believe is a stronger manuscript.
\\ \\ \noindent 
We also note that a few typos from the manuscript’s original form have
been addressed. An itemized list of the changes addressing the
reviewers’ comments are below.
\\ \\ \noindent 
Sincerely, \\ \noindent
Yuan-Nan Young, Shravan Veerapaneni, and Bryan Quaife

\vspace{20pt}
\noindent
\comment{General comment: I found that the figure captions are often
incomplete and that details of the legends are missing and are only
given in the text. I think figures should be self explanatory (e.g.
signification of symbols of fig. 9 right (angles corresponding to symbol
sizes); in fig. 10: scale/unit of vorticity, Fig 11: indicate in the
caption that the coloraturas corresponds to tension; Fig 12: color code
in figure caption...)}
\begin{itemize}
  \item rebuttal
\end{itemize}

\noindent
\comment{Abstract: it should be mentioned in the abstract that this is
2D modelling. 2D simulations of vesicles are often perfectly fine to
investigate the physics of phenomena but in some cases 3D effects may
have their importance, at least for quantitative comparisons with
experiments. The reader should identify either from the title or
abstract that this study is 2D.}
\begin{itemize}
  \item We now mention in the abstract that we consider the dynamics of
    and adhesive two-dimensional vesicle doublet.
\end{itemize}

\noindent
\comment{Introduction, 2nd paragraph: the transition in the effective
viscosity of vesicle or RBC suspensions was also validated
experimentally in ref [18] (Vitkova et al.), not only in [20-22].}
\begin{itemize}
  \item We have included reference [18] in the list of experimental
    works that validated the transition in the effective viscosity
\end{itemize}

\noindent
\comment{Intro, 3rd paragraph: the buckling instability of aggregating RBC
membranes was not only predicted theroretically and numerically in
Flormann et al. [32] but also experimentally. In addition, a more
recent paper by Hoore et al. (Soft Matter 14, 6278-6289 (2018)) showed
the influence of the membrane shear elasticity on doublet shapes by 3D
modelling. This reference may be added.}
\begin{itemize}
  \item rebuttal
\end{itemize}

\noindent
\comment{Concerning the stability and dynamics of clusters of vesicles with
adhesive interaction, there have been at least two papers on clusters
in micro-channel flow (computational and experimental) that should be
cited: Brust et al., Scientific Reports 4, 4348 (2014) and Claveria et
al., Soft Matter. 2016 vol 12(39):8235-8245. About rheology, at least
the historical paper by Chien on the shear-thinning of blood related
to RBC aggregation should be cited : Chien, S., Usami, S., Dellenback,
R.~J., Gregersen, M.~I., Nanninga, L.~B.~\& Guest, M.~M. Blood
viscosity: Influence of erythrocyte aggregation. Science 157, 829–831
(1967).}
\begin{itemize}
  \item rebuttal
\end{itemize}

\noindent
\comment{In addition to ref [54], Gires et al. also published experimental
measurements of hydrodynamic interactions between vesicles in Gires et
al., Phys.~Fluids 26, 013304 (2014)}
\begin{itemize}
  \item rebuttal
\end{itemize}

\noindent
\comment{In II. Governing equations and fig. 1: the right panel of the
figure and comments at the end of the section say that the part of
vesicle membranes that is out of the contact zone, including the parts
that are facing the outer fluid, experience an attractive interaction.
This is an artifact due to the long-range potential chosen by the
authors to model the interaction. In most experimental systems, the
adhesive forces leading to cluster formation are either due to direct
bridging of membranes or depletion phenomena that lead to osmotic
pressure gradients. This is the case for RBC aggregation for instance.
In both mechanisms, there is indeed no attractive force between membrane
parts that are out of the contact zone. It is perfectly fine to model
the interaction by an adhesion potential such as represented in fig 1
left, but its use introduces this artifact that should be acknowledged
by the authors. It is also due to the fact that the value of parameter
$\delta$ (the equilibrium distance) is not small compared to vesicle
dimensions. This is a numerical necessity to avoid the need to have
extremely refined meshes but again in experiments, the equilibrium
distance is usually much smaller that vesicle sizes and is more
comparable to molecular sizes (size of bridging molecules or of
depletion-inducing macromolecules). This deserves to be acknowledged
also, even though it may not impact too much the results.}
\begin{itemize}
  \item rebuttal
\end{itemize}

\noindent
\comment{Along the same line, the repulsing spots in fig 1 right or Fig 3
right as well as the amplitude of the deformation in the contact zone
are again a consequence of the choice of parameter values for the
interaction potential and would probably not be so strong in real
cases where the equilibrium distance between membranes is much shorter
and the range of the potential is shorter.}
\begin{itemize}
  \item rebuttal
\end{itemize}

\noindent
\comment{I cannot figure out why the external part of vesicles on profile 1
in fig 3-left is yellow. Shouldn't the force be weakly attractive in
that region, as in fig 1-right?}
\begin{itemize}
  \item rebuttal
\end{itemize}

\noindent
\comment{Figs 4-5 and related text: it would have been interesting to
compare results with the bending energy of vesicles without adhesion
potential in their equilibrium shape, to show the energy gain due to
adhesion (bending and total energy). This could be a way to compare
the total energy of the doublet to the energy of dissociated vesicles
and infer about the stability of doublets.}
\begin{itemize}
  \item rebuttal
\end{itemize}

\noindent
\comment{Fig 9(right): I regret that the authors did not extend the study
to lower values of the reduced area, to reach values comparable to red
blood cells which are more deflated that $\Delta A=0.7$. Also, the range
of shear rates could have been slightly increased to reach the
stability limit of the most deflated vesicles. Here, we do not know
what happens below reduced areas of 0.8.}
\begin{itemize}
  \item We have extended the results to include vesicles of reduced area
    0.65, a typical value for two-dimensional red blood cells.
  \item We have also reported results for larger extensional rates so
    that the transition from a fluid trap to two separating vesicles can
    be observed at all the reduced areas.
\end{itemize}

\noindent
\comment{ Figs 9(right) and 12(right): the stability diagrams are presented
in different parameter spaces for the extensional and shear flow
cases. It would have been interesting to see e.g. the diagram in the
interaction energy vs. flow strength $\chi$ for both cases, for
comparison.}
\begin{itemize}
  \item rebuttal
\end{itemize}

\noindent
\comment{Fig.~10: the black arrows, as I understand, represent the total
attractive force due to the interaction potential. Would the authors
be able to represent also the total hydrodynamic forces on vesicles
which presumably balance the torque of interaction forces? Is there a
tank-treading motion of membranes?}
\begin{itemize}
  \item rebuttal
\end{itemize}

\noindent
\comment{Fig 12: the authors did not comment the evolution of the
frequency of distance oscillations in doublets when the Hamaker constant
is varied. Can it be compared to the natural rotation frequency of the
shear flow $\chi/2$? Does a strongly adhesive doublet tend to behave as
a tumbling rigid object following something similar to Jeffery orbits?}
\begin{itemize}
  \item We now report the period of the doublets at three different
    shear rates and at a variety of Hamaker constants.

  \item For each of the shear rates and the largest Hamaker constant we
    consider, we compute an effective aspect ratio which is the aspect
    ratio of a rigid body ellipse that would undergo the same period in
    the same shear flow.  The results indicate that the effective aspect
    ratio is independent of the shear rate, and this suggests that the
    dynamics of the doublet has similar properties to a rigid body
    undergoing a Jeffery orbit. 
\end{itemize}

\noindent
\comment{Fig 12-right: is the stability limit between the red and blue
zones linear? This would mean that a stability criterion could take
the form of a dimensionless number involving the ratio of Hamaker
constant/shear rate, a potentially interesting result.}
\begin{itemize}
  \item We thank the reviewer for this observation.  We increased the
    parameter space of the phase diagram, and it does appear that there
    is a linear relationship.
  \item We have superimposed a straight line to show the transition
    between the two doublet and unbounded configurations.
  \item We also have repeated the phase diagram, but with a smaller
    separation distance.  With this new separation distance, the
    transition between the two regions appears to also be linear.
\end{itemize}

\noindent
\comment{Fig 13 and comments about rheology: I am not sure to understand
the meaning of the left figure where it is shown that the intrinsic
viscosity of the doublet in suspension is about twice the viscosity of a
single vesicle. By definition, the intrinsic viscosity does not depend
on volume fraction, so I do not expect such a big difference between
bound vesicles and isolated vesicles. With the values chosen for the
Hamaker constant and the shear rate, it seems (from fig 12-left) that
the interaction is quite strong and that the distance between the two
vesicles in the doublet is always very short (close to the equilibrium
distance $\delta$). So the doublet must be roughly tumbling as a single
object and unless a co-rotating tank-treading motion of the two
membranes produces a strong dissipation by shear flow in the gap between
the two vesicles, the difference should not be so large. It is also
quite curious that the intrinsic viscosity of the doublet is so close to
exactly twice the viscosity of a single vesicle. Could the authors check
and elaborate on this ? Was the volume fraction properly scaled out
here? Also in the right panel, the bending/tension part is compared to
twice the intrinsic viscosity of a suspension of isolated vesicles. Why
twice ? Finally, it is not clear how technically the bending/tension and
adhesion parts are separated.  Overall I think the study on rheology
needs much more development than what is presented here with only one
set of parameter values and very little quantitative comments.}
\begin{itemize}
  \item The reviewer is correct that we miscalculated the intrinsic
    viscosity.  We had initially divided by only the area inside a
    single vesicle rather than both vesicles in the doublet.  This has
    been corrected, and now there is only a small increase in the
    intrinsic viscosity when compared to a single tank treading vesicle.

  \item With the numbers corrected, the intrinsic viscosity is clearly
    not equal to the intrinsic viscosity of a tank treading vesicle.

  \item We have double-checked our calculations and the volume fraction
    has been properly accounted for.

  \item Using twice the intrinsic viscosity of a tank treading vesicle
    was incorrect.  We now compare the intrinsic viscosity of a single
    tank treading vesicle with the doublet.

  \item The total force $\xxi$ is the sum of a bending, tension, and
    adhesion force.  Therefore, the contribution to the intrinsic
    viscosity due to each of these forces can be individually computed.
\end{itemize}

\noindent
\comment{In the conclusions, an example of aggregation of vesicles enclosed
in a bigger one is presented. I do not think this brings much to this
paper. Why not keep this for a future paper dedicated to this specific
problem and instead show examples of aggregates without the enclosing
large vesicle? In what conditions (reduced volume, Hamaker constant)
can linear structures (rouleaux) be formed for instance (see the paper
by Hoore et al whose reference is given above for instance).}
\begin{itemize}
  \item We have removed this example from the manuscript.  Instead, we
    mention that future work includes a suspension of many vesicles.
\end{itemize}




\end{document}
