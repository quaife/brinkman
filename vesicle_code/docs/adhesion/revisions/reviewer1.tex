\documentclass[11pt]{article}

\usepackage{fullpage}
\usepackage{todonotes}
\usepackage{amsmath,amsfonts,amssymb,stmaryrd}
\usepackage{color}
\newcommand{\comment}[1]{{\color{blue} #1}}

\begin{document}
\noindent
Thank you for your constructive comments.  The attached manuscript
addresses all the reviewer 1's comments and the result is what we
believe is a stronger manuscript.
\\ \\ \noindent 
We also note that a few typos from the manuscript’s original form have
been addressed. An itemized list of the changes addressing the
reviewers’ comments are below.
\\ \\ \noindent 
Sincerely, \\ \noindent
Yuan-Nan Young, Shravan Veerapaneni, and Bryan Quaife

\vspace{20pt}
\noindent
\comment{While the numerical simulations performed are based on standard
approaches, and the results are supposedly correct, I find the results
not to have a significant impact. Most results presented, for example,
those on vesicle adhesion in quiescent flow, adhered vesicle dynamics in
shear flow, and adhesion of two approaching vesicles in a shear flow,
are not surprising, and similar results were obtained before for red
blood cells, capsules, drops, and adhered particles. Most (maybe all) of
the principals discussed are more-or-less understood, and would
therefore not of themselves constitute a novel contribution.}
\begin{itemize}
  \item We thank the referee for this comment. The referee is correct by labeling boundary integral simulation as a standard approach.
  However, the boundary integral algorithm we used for this work is not standard boundary integral algorithm for two reasons: (1) First the time marching is adaptive, allowing us to efficiently run the simulations for a long time with accuracy and convergence. (2) Second we adopted a slightly sophisticated near singular evaluation to ensure that our numerical results are convergent when two boundaries are close to each other. Without adaptive time marching, the boundary integral simulations of vesicles often give unreliable results as shown in
Kabacaoglu {\it et al.}, J. Comp. Phys. (2018). In addition it is very time consuming to simulate two vesicles to equilibrate in close vicinity of each other in a quiescent flow. 
Without a good near-singular evaluation in the boundary integral calculation, the simulations will be contaminated by the errors incurred when two moving deformable boundaries are close to each other.
  
  We thank the referee for pointing out that similar physical setup has been studied for red blood cells, capsules, drops and adhere particles. 
  However it is not clear to us if our results for vesicles are so similar to those for red blood cell/drop/particles in the literature that no new understanding is contributed in this work.
  
For example, we are able to verify the scaling of the draining time with respect to adhesion strength and equilibrium separation distance by analyzing the numerical simulation results. 
A similar scaling is found in Ramachandran {\it et al.}, Phys. Fluids (2010) where they assumed the vesicles are spherical cap with a flat contact region. It is not clear how their scaling for the draining time can differentiate the contribution from either adhesion strength or equilibrium separation distance, while our scaling clearly illustrates how each component contributes differently to the draining time. It is also not clear how such scaling can be verified by comparing against experimental data.

For the vesicles in a planar extensional flow, we cannot find any work in the literature that shows how two red blood cells/capsules move toward each other around the stagnation point. Furthermore we cannot find any work in the literature that uses extensional flow to measure the adhesion strength between two interfaces/particles/membranes. All the work we can find on drop in an extensional flow/four-roll mill is for a single drop (clean drop or a surfactant-laden drop).
  
  
\end{itemize}

\noindent
\comment{Second, the simulations are 2D, which makes it difficult to accept
whether the predicted results would be observed in experiments. There
was also no direct comparison with any experimental studies.}
\begin{itemize}
  \item rebuttal
\end{itemize}

\noindent
\comment{My third concern is the use of L-J potential. In practice, the
vesicle surface would be coated with some adhesive molecules (e.g.
receptor/ligand). Is the L-J type adhesive field appropriate for such
interactions? ALso, in reality, these bonds are stochastic in nature.}
\begin{itemize}
  \item rebuttal
\end{itemize}

\noindent
\comment{The authors talks about the "fluid trap". This is indeed a tricky
issue in any experiment dealing with extensional flow as how to
maintain the drop/vesicle at the stagnation point. But in a numerical
simulation this should not be any issue at all., because even of the
vesicle moves away from the stagnation point its dynamics is still
governed by the velocity gradient which is same everywhere in the flow
field.}
\begin{itemize}
  \item rebuttal
\end{itemize}

\noindent
\comment{There was not enough physical insights on the vesicle alignment
in extensional flow as presented in section IV. If the vesicles are
placed centrally, what causes the pair to become asymmetric in the first
place, since all components in the model are deterministic? The authors
used vorticity contours, but the external flow is irrotational; so what
is the origin of vorticity, and is it appropriate to use vorticity to
explain the behavior?}
\begin{itemize}
  \item rebuttal
\end{itemize}

\noindent
\comment{I did not find figure 14 relevant for the present article. This
result is not relevant, and also presented just to show what the authors
intend to do in future.}
\begin{itemize}
  \item We have removed this example from the manuscript.  Instead, we
    mention that future work includes a suspension of many vesicles.
\end{itemize}



\end{document}
