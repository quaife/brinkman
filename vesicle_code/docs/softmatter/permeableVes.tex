%%%%%%%%%%%%%%%%%%%%%%%%%%%%%%%%%%%
%This is the LaTeX ARTICLE template for RSC journals
%Copyright The Royal Society of Chemistry 2016
%%%%%%%%%%%%%%%%%%%%%%%%%%%%%%%%%%%

\documentclass[twoside,twocolumn,9pt]{article}
\usepackage{extsizes}
\usepackage[super,sort&compress,comma]{natbib} 
\usepackage[version=3]{mhchem}
\usepackage[left=1.5cm, right=1.5cm, top=1.785cm, bottom=2.0cm]{geometry}
\usepackage{balance}
\usepackage{times,mathptmx}
\usepackage{sectsty}
\usepackage{graphicx} 
\usepackage{lastpage}
\usepackage[format=plain,justification=justified,singlelinecheck=false,font={stretch=1.125,small,sf},labelfont=bf,labelsep=space]{caption}
\usepackage{float}
\usepackage{fancyhdr}
\usepackage{fnpos}
\usepackage[english]{babel}
\addto{\captionsenglish}{%
  \renewcommand{\refname}{Notes and references}
}
\usepackage{array}
\usepackage{droidsans}
\usepackage{charter}
\usepackage[T1]{fontenc}
\usepackage[usenames,dvipsnames]{xcolor}
\usepackage{setspace}
\usepackage[compact]{titlesec}
\usepackage{hyperref}
%%%Please don't disable any packages in the preamble, as this may cause the template to display incorrectly.%%%


\usepackage{epstopdf}%This line makes .eps figures into .pdf - please comment out if not required.

\definecolor{cream}{RGB}{222,217,201}

\usepackage{amsmath,amsfonts,amssymb,bm}
\usepackage{epsfig}
\DeclareMathAlphabet{\mathcal}{OMS}{cmsy}{m}{n}

\input my_macros2.tex	
\begin{document}

\pagestyle{fancy}
\thispagestyle{plain}
\fancypagestyle{plain}{

%%%HEADER%%%
\fancyhead[C]{\includegraphics[width=18.5cm]{head_foot/header_bar}}
\fancyhead[L]{\hspace{0cm}\vspace{1.5cm}\includegraphics[height=30pt]{head_foot/journal_name}}
\fancyhead[R]{\hspace{0cm}\vspace{1.7cm}\includegraphics[height=55pt]{head_foot/RSC_LOGO_CMYK}}
\renewcommand{\headrulewidth}{0pt}
}
%%%END OF HEADER%%%

%%%PAGE SETUP - Please do not change any commands within this section%%%
\makeFNbottom
\makeatletter
\renewcommand\LARGE{\@setfontsize\LARGE{15pt}{17}}
\renewcommand\Large{\@setfontsize\Large{12pt}{14}}
\renewcommand\large{\@setfontsize\large{10pt}{12}}
\renewcommand\footnotesize{\@setfontsize\footnotesize{7pt}{10}}
\makeatother

\renewcommand{\thefootnote}{\fnsymbol{footnote}}
\renewcommand\footnoterule{\vspace*{1pt}% 
\color{cream}\hrule width 3.5in height 0.4pt \color{black}\vspace*{5pt}} 
\setcounter{secnumdepth}{5}

\makeatletter 
\renewcommand\@biblabel[1]{#1}            
\renewcommand\@makefntext[1]% 
{\noindent\makebox[0pt][r]{\@thefnmark\,}#1}
\makeatother 
\renewcommand{\figurename}{\small{Fig.}~}
\sectionfont{\sffamily\Large}
\subsectionfont{\normalsize}
\subsubsectionfont{\bf}
\setstretch{1.125} %In particular, please do not alter this line.
\setlength{\skip\footins}{0.8cm}
\setlength{\footnotesep}{0.25cm}
\setlength{\jot}{10pt}
\titlespacing*{\section}{0pt}{4pt}{4pt}
\titlespacing*{\subsection}{0pt}{15pt}{1pt}
%%%END OF PAGE SETUP%%%

%%%FOOTER%%%
\fancyfoot{}
\fancyfoot[LO,RE]{\vspace{-7.1pt}\includegraphics[height=9pt]{head_foot/LF}}
\fancyfoot[CO]{\vspace{-7.1pt}\hspace{13.2cm}\includegraphics{head_foot/RF}}
\fancyfoot[CE]{\vspace{-7.2pt}\hspace{-14.2cm}\includegraphics{head_foot/RF}}
\fancyfoot[RO]{\footnotesize{\sffamily{1--\pageref{LastPage} ~\textbar  \hspace{2pt}\thepage}}}
\fancyfoot[LE]{\footnotesize{\sffamily{\thepage~\textbar\hspace{3.45cm} 1--\pageref{LastPage}}}}
\fancyhead{}
\renewcommand{\headrulewidth}{0pt} 
\renewcommand{\footrulewidth}{0pt}
\setlength{\arrayrulewidth}{1pt}
\setlength{\columnsep}{6.5mm}
\setlength\bibsep{1pt}
%%%END OF FOOTER%%%

%%%FIGURE SETUP - please do not change any commands within this section%%%
\makeatletter 
\newlength{\figrulesep} 
\setlength{\figrulesep}{0.5\textfloatsep} 

\newcommand{\topfigrule}{\vspace*{-1pt}% 
\noindent{\color{cream}\rule[-\figrulesep]{\columnwidth}{1.5pt}} }

\newcommand{\botfigrule}{\vspace*{-2pt}% 
\noindent{\color{cream}\rule[\figrulesep]{\columnwidth}{1.5pt}} }

\newcommand{\dblfigrule}{\vspace*{-1pt}% 
\noindent{\color{cream}\rule[-\figrulesep]{\textwidth}{1.5pt}} }

\makeatother
%%%END OF FIGURE SETUP%%%

%%%TITLE, AUTHORS AND ABSTRACT%%%
\twocolumn[
\begin{@twocolumnfalse}
\vspace{3cm}
\sffamily
\begin{tabular}{m{4.5cm} p{13.5cm} }

\includegraphics{head_foot/DOI} & \noindent\LARGE{\textbf{Semi-permeable
  vesicles: Mathematical formulation, numerics, and physics}} \\
\vspace{0.3cm} & \vspace{0.3cm} \\

 & \noindent\large{Ashley Gannon,\textit{$^{a}$} Yuan-Nan
  Young,\textit{$^{b}$} Shravan Veeranapanei,\textit{$^{c}$} and Bryan
  Quaife$^{\ast}$ \textit{$^{d}$}} \\

\includegraphics{head_foot/dates} & 
\noindent\normalsize{We consider a single semi-permeable vesicle in a
  variety of flows} \\
\end{tabular}

\end{@twocolumnfalse} \vspace{0.6cm}

]
%%%END OF TITLE, AUTHORS AND ABSTRACT%%%

%%%FONT SETUP - please do not change any commands within this section
\renewcommand*\rmdefault{bch}\normalfont\upshape
\rmfamily
\section*{}
\vspace{-1cm}


%%%FOOTNOTES%%%

\footnotetext{\textit{$^{a}$~Department of Scientific Computing, Florida
State University, Tallahassee, FL, USA.}}
\footnotetext{\textit{$^{b}$~Department of Mathematical Sciences, New
Jersey Institute of Technology, Newark, NJ, USA.}}
\footnotetext{\textit{$^{c}$~Department of Mathematics, University of
Michigan, Ann Arbor, MI, USA.}}
\footnotetext{\textit{$^{d}$~Department of Scientific Computing, Florida
State University, Tallahassee, FL, USA. E-mail: bquaife@fsu.edu}}

%Please use \dag to cite the ESI in the main text of the article.
%If you article does not have ESI please remove the the \dag symbol from the title and the footnotetext below.
%\footnotetext{\dag~Electronic Supplementary Information (ESI) available:
%[details of any supplementary information available should be included
%here]. See DOI: 00.0000/00000000.}
%\footnotetext{\ddag~Additional footnotes to the title and authors can be
%included \textit{e.g.}\ `Present address:' or `These authors contributed
%equally to this work' as above using the symbols: \ddag, \textsection,
%and \P. Please place the appropriate symbol next to the author's name
%and include a \texttt{\textbackslash footnotetext} entry in the the
%correct place in the list.}


%%%END OF FOOTNOTES%%%

%%%MAIN TEXT%%%%
%%%%%%%%%%%%%%%%%%%%%%%%%%%%%%%%%%%%%%%%%%%%%%%%%%%%%%%%%%%%%%%%%%%%%%%%
\section{Introduction}

%%%%%%%%%%%%%%%%%%%%%%%%%%%%%%%%%%%%%%%%%%%%%%%%%%%%%%%%%%%%%%%%%%%%%%%%
\section{Problem formulation} \label{sc:formulate}
Consider a semi-permeable vesicle suspended in an unbounded viscous
fluid domain, subjected to an imposed flow $\mbu_\infty(\mbx)$, for any
$\mbx \in \RR^2$.  Assume that the interior and exterior fluids have the same viscosity $\mu$. Let $\mbx$ be the position of the interface $\gamma$, $\mbu$ the fluid velocity and $p$ the pressure. In the vanishing Reynolds number limit, the governing equations for the ambient fluid can be written as: 
%
\begin{subequations}
\begin{align}
-\nabla p + \mu \triangle \mbu = 0 \quad\text{in}\quad
  \RR^2\setminus\gamma  &  \\
% 
\nabla \cdot \mbu = 0  \quad\text{in}\quad \RR^2\setminus\gamma &\\
%
\mbu(\mbx) \rightarrow \mbu_\infty(\mbx) \quad \text{as} \quad  \norm{\mbx} \rightarrow \infty & 
 \end{align} \label{eqn:governing}
\end{subequations}
\begin{figure}[H]
	\centering
	\epsfig{file =figures/vesicleschem.pdf, width=.6\columnwidth}
	\caption{$\Omega$ is the unbounded fluid domain and $\gamma$ is the vesicle membrane. In addition to the vesicle-induced flow, a shear flow is imposed in the far field, $\mbu_\infty$.}
\end{figure}
The modified jump conditions in the case of a semi-permeable vesicle can
be summarized as follows.  Let $\mbn$ the outward normal to $\gamma$ and
$\jump{\cdot}$ denote the jump across the interface. The jump condition
for the hydrodynamic stress remains the same:
\begin{align} 
  \jump{\Sigma\cdot\mbn} = \mbf, 
\end{align}
where $\Sigma$ is the hydrodynamic stress tensor. The fluid velocity is
continuous across the membrane i.e.,
\begin{align}
  \jump{\mbu} = 0. 
\end{align}
However, the kinematic condition is modified to account for the fact
that the fluid can move through the membrane. A standard choice appears
to be
\begin{align}
  \mbu - \dot{\mbx} = - \beta (\text{RT}\jump{c}+\mbf \cdot \mbn) \mbn
      \quad\text{on}\quad\gamma. 
\end{align}
Here, $\beta$, R, and T are constants, $c$ is the solute concentration, and the membrane force is the usual sum of the bending and tension forces,
\begin{subequations}
	\begin{align}
	\mbf = \mbf_{\text{bending}}+\mbf_{\text{tension}},  &  \\
	% 
	\mbf_{\text{bending}} = -\kappa_b\mbx_{ssss}, &\\
	%
	\mbf_{\text{tension}} = (\sigma\mbx_s)_s. & 
	\end{align} \label{eqn:bendten}
\end{subequations}
%%%%%%%%%%%%%%%%%%%%%%%%%%%%%%%%%%%%%%%%%%%%%%%%%%%%%%%%%%%%%%%%%%%%%%%%
\section{Integral equation formulation and non-dimensionalization}
There are multiple ways to convert the PDE problem into integral
equations. The coupled set of
integro-differential equations satisfy the given interfacial conditions:
\begin{align}
  \label{eqn:vesVelocity}
  \dot{\mbx} = \mbu_\infty(\mbx) + \beta
  (RT\jump{c}+\mbf\cdot\mbn)\mbn
  + \sgl [\mbf], & \\
  \mbx_s\cdot\dot{\mbx}_s = 0 &
\end{align}
This satisfies all the jump conditions. As usual, $\sgl$ is the single
layer potential.  We introduce the notation
\begin{align}
  B[\mbx]\mbf &= -\kappa_b\frac{d^4}{ds^4} \mbf, \\
  T[\mbx]\sigma &= (\sigma \mbx_s)_s, \\
  P[\mbx]\mbf &= (\mbf \cdot \mbn) \mbn, \\
  D[\mbx]\mbf &= \mbx_s \cdot \mbf_s,
\end{align}
where $\mbn$ is the outward unit normal of the vesicle parameterized by
$\mbx$, and $d/ds$ is its arclength derivative.  Dropping the dependence
of the operators on $\mbx$, the dynamics of the vesicle are governed by
\begin{align}
  \dot{\mbx} &= \mbu_{\infty}(\mbx) + 
  \beta RT\jump{c}\mbn + \beta P(B\mbx + T\sigma) + \sgl(B\mbx + T\sigma), \label{eq:full}\\
  D \dot{\mbx} &= 0.
\end{align}

Following the analysis of~\cite{vee-gue-zor-bir2009} and
disregarding background flow, we define scales of $\mbx$, $t$, $\beta$,
and $c$ to nondimensionalize equation (\ref{eq:full}) as follows. The
length is scaled using the radius of a circle with the same perimeter,
$L$, as the vesicle, $R_0 = \frac{L}{2\pi}$. From analysis, the time
scale $\tau$ is defined as $\tau = \frac{\mu R_0^2}{k_b}$, the tension
scale, $Q$, is defined as $Q = \frac{k_b}{R_0^2}$, the water flux scale,
$W$, is defined as $W=\frac{R_0}{\mu}$, and the concentration scale,
$C$, is defined as $C=\frac{k_b}{R_0^3RT}$. The governing equation in
nondimensional form is
\begin{align}
    \dot{\Tilde{\mbx}} = \Tilde{\mbu}_{\infty} + \Tilde{\beta}\jump{\tilde{c}}\mbn +
    \Tilde{\beta}P\big(B[\Tilde{\mbx}]\Tilde{\mbx}+T[\Tilde{\mbx}]\Tilde{\sigma}\big)
    +\sgl\big( B[\Tilde{\mbx}]\Tilde{\mbx} + T[\Tilde{\mbx}]\Tilde{\sigma}\big).
\end{align}
In our current implementation, we let $\jump{c} = 0$. The remaining equations are written in dimensionless form, unless stated otherwise, and for simplicity of notation we remove the tildes.


%%%%%%%%%%%%%%%%%%%%%%%%%%%%%%%%%%%%%%%%%%%%%%%%%%%%%%%%%%%%%%%%%%%%%%%%
\section{Numerical methods}
To address stiffness associated with high-order derivatives,
equation~\eqref{eqn:vesVelocity} is discretized with a semi-implicit
time stepping method similar to our previous works.  
where $\mbn$ is the outward unit normal of the vesicle
parameterized by $\mbx$, and $d/ds$ is its arclength derivative.
Dropping the dependence of the operators on $\mbx$, the dynamics of the
vesicle are governed by
\begin{align}
  \dot{\mbx} &= \mbu_{\infty}(\mbx) + 
  \beta P(B\mbx + T\sigma) + \sgl(B\mbx + T\sigma), \\
  D \dot{\mbx} &= 0.
\end{align}
Using the notation $B^N$ to denote the bending operator due to the
vesicle configuration at time $t_N$, and similar notation for the other
operators, we apply the semi-implicit time stepping method
\begin{align} 
\begin{split}
  \frac{\mbx^{N+1} - \mbx^N}{\Delta t} = \mbu_\infty(\mbx^N) 
  + \beta P^N(B^N\mbx^{N+1} + T^N\sigma^{N+1}) \\
   + \sgl^N(B^N\mbx^{N+1} + T^N\sigma^{N+1}).
   \end{split} 
\end{align}
Applying $D$ to the first-order time discretization of
equation~\eqref{eqn:vesVelocity}, the inextensibility condition is
\begin{align*}
  &D^N(\mbx^{N+1} - \mbx^N) = 0 \\ 
  \Rightarrow &D^N\mbx^{N+1} = D^N \mbx^N \\ 
  \Rightarrow &D^N\mbx^{N+1} = 1.
\end{align*}
Finally, we write the time stepping method in matrix form as
\begin{align}
  A \left(
    \begin{array}{c}
      \mbx^{N+1} \\ \sigma^{N+1}
    \end{array}
  \right) = 
  \left(
    \begin{array}{c}
      \mbx^{N} + \Delta t \mbu_\infty(\mbx^N) \\ \mathbf{1}
    \end{array}
  \right),
\end{align}
where
\begin{align}
 A = \left(
  \begin{array}{cc}
    I + \Delta t \beta P^N B^N + \Delta t \sgl^N B^N & 
    -\Delta t P^N T^N - \Delta t \sgl^N T^N \\
    D^N & 0
  \end{array}
  \right).
\end{align}

Analytically, we use the governing equations to write a differential equation for
the area of the vesicle as a function of time. The area inside a single
vesicle $\gamma$ parameterized with $\mbx$ is
\begin{align}
A(t) = \frac{1}{2}\int_{\gamma} (\mbx \cdot \mbn)\, ds = 
\frac{1}{2}\int_{0}^{2\pi} \left(\mbx \cdot
\pderiv{\mbx}{\theta}^\perp\right)
\, d\theta.
\end{align}
Taking the time derivative of $A(t)$,
\begin{align}
\dot{A}(t) &= \frac{1}{2}\int_{0}^{2\pi} \left(\dot{\mbx} \cdot
\pderiv{\mbx}{\theta}^{\perp} + 
\mbx \cdot \pderiv{\dot{\mbx}}{\theta}^{\perp}\right) d\theta, \\
%&= \frac{1}{2}
%\int_{0}^{2\pi}(\dot{\mbx}\cdot\mbn)\left\|\pderiv{\mbx}{\theta}\right\|d\theta +
%   \frac{1}{2}\int_{0}^{2\pi}
%  \left(\mbx \cdot \pderiv{\dot{\mbx}}{\theta}^{\perp}\right)d\theta, \\
%&= \frac{\beta}{2} \int_{\gamma}\left(\mbf\cdot\mbn\right)ds +
%\frac{1}{2}\int_{0}^{2\pi}\left(\mbx \cdot \pderiv{}{t}\left(\mbn
%\left\|\pderiv{\mbx}{\theta}\right\|\right)\right)d\theta, \\
%&= \frac{\beta}{2} \int_{\gamma}\left(\mbf\cdot\mbn\right)ds + \frac{1}{2}
%\int_{\gamma}\left(\mbx\cdot\dot{\mbn}\right)ds + \frac{1}{2}\int_{0}^{2\pi}
%\frac{\mbx\cdot\mbn}{\left\|\pderiv{\mbx}{\theta}\right\|}
%\left(\pderiv{\mbx}{\theta}\cdot
%\pderiv{\dot{\mbx}}{\theta}\right)d\theta \\
&= \frac{\beta}{2} \int_{\gamma}\left(\mbf\cdot\mbn\right)ds + \frac{1}{2}
\int_{\gamma}\left(\mbx\cdot\dot{\mbn}\right)ds + \frac{1}{2}\int_{\gamma}
(\mbx\cdot\mbn) (\mbx_s \cdot \dot{\mbx}_s)\, ds.
\end{align}
To satisfy the inextensibilty condition, $\mbx_s \cdot \dot{\mbx}_s = 0$, the time derivative of $A(t)$ is
\begin{align}
\dot{A}(t) &= \frac{\beta}{2}
\int_{\gamma}\left(\mbf\cdot\mbn\right)ds
+ \frac{1}{2} \int_{\gamma}\left(\mbx\cdot\dot{\mbn}\right)ds.
\end{align}
%... 
%\begin{align}
%    \dot{A}(t) &= \frac{\beta}{2}
%    \int_{\gamma}\left(\mbf\cdot\mbn\right)ds+\frac{1}{2}\int_{\gamma}\left(\mbx\cdot\dot{\mbn}\right) ds,
%\end{align}
which we re-write as
\begin{align}
\dot{A}(t) &= \frac{\beta}{2}
\int_{\gamma}\left(\mbf\cdot\mbn\right)ds-\frac{1}{2}\int_{\gamma}\left(\mbx^{\perp}\cdot\dot{\mbx}_s\right) ds.
\end{align}
Applying integration by parts and re-writing,
%\begin{align}
%    \dot{A}(t) &= \frac{\beta}{2}
%    \int_{\gamma}\left(\mbf\cdot\mbn\right)ds+\frac{1}{2}\int_{\gamma}\left(\mbx^{\perp}_s\cdot\dot{\mbx}\right) ds,
%\end{align}
%which we re-write as
\begin{align}
%   \dot{A}(t) &= \frac{\beta}{2}
%   \int_{\gamma}\left(\mbf\cdot\mbn\right)ds+\frac{1}{2}\int_{\gamma}\left(\dot{\mbx}\cdot\mbn\right),\\
\dot{A}(t) &= \beta \int_{\gamma}\left(\mbf\cdot\mbn\right)ds.
\end{align}
At steady state,
\begin{align}
\dot{A}(t) &= \beta \int_{\gamma}\left(\mbf\cdot\mbn\right)ds =0.
\end{align}
Replacing $\mbf$ with its relation, applying integration by parts twice and using Frenet-serret formulas,  
\begin{align}
\beta\int_{\gamma}\kappa_b\kappa^3 ds +\beta\int_{\gamma} \kappa\sigma ds  = 0. 
\end{align} 
%We now replace $\mbf$ with its relation
%\begin{align}
%   \beta\int_{\gamma}
%   \left((-\kappa_b\mbx_{ssss}+(\sigma\mbx_s)_s)\cdot\mbn\right)ds &=0.
%\end{align}
%Applying integration by parts,
%\begin{align}
%     \beta\int_{\gamma}\left(\kappa_b\mbx_{sss}-\sigma\mbx_s\right)\cdot
%     \mbn_s ds &=0.
%\end{align}
%From the Frenet-Serret formulas, we have 
%\begin{align}
%     -\beta\int_{\gamma}
%     \left(\kappa_b\mbx_{sss}-\sigma\mbx_s\right)\cdot\kappa\mbx_s ds &=0.
%\end{align}

%Applying integration by parts and Frenet-Serret again
%\begin{align}
%     \beta\int_{\gamma}
%     \kappa_b\kappa\left(\mbx_{ss}\cdot\mbx_{ss}\right)ds + \beta
%     \int_{\gamma} \kappa_b \kappa_s(\mbx_{ss} \cdot \mbx_s)ds +\beta\int_{\gamma}\kappa\sigma ds  &= 0,\\
%    \beta\int_{\gamma} \kappa_b\kappa(\kappa\mbn \cdot
%     \kappa\mbn)+ \beta \int_{\gamma} \kappa_b \kappa_s(\mbx_{ss} \cdot
%     \mbx_s)ds +\beta\int_{\gamma}\kappa\sigma ds &= 0,
%\end{align}
%which we re-write as
%\begin{align}
%   \beta\int_{\gamma}\kappa_b\kappa^3 ds + \beta \int_{\gamma} \kappa_b
%   \kappa_s(\mbx_{ss} \cdot \mbx_s)ds +\beta\int_{\gamma} \kappa\sigma ds = 0.
%\end{align}
%Applying Frenet-serret again,
%\begin{align}
%    \beta\int_{\gamma}\kappa_b\kappa^3 ds + \beta \int_{\gamma} \kappa_b
%    \kappa_s(\kappa\mbn \cdot \TT)ds +\beta\int_{\gamma} \kappa\sigma ds = 0, \\
%    \beta\int_{\gamma}\kappa_b\kappa^3 ds +\beta\int_{\gamma} \kappa\sigma ds  = 0, 
%\end{align}
%we compare this to our numerical results.



%%%%%%%%%%%%%%%%%%%%%%%%%%%%%%%%%%%%%%%%%%%%%%%%%%%%%%%%%%%%%%%%%%%%%%%%
\section{Numerical simulations}
We consider a single vesicle suspended in a viscous fluid under various
conditions.  The main results we have so far are
\begin{itemize}
  \item Given a fixed $\beta$ and vesicle length, the final reduced area
    is independent of the initial reduced area, and even the vesicle
    shape. However, the final reduced area does depend on the initial
    vesicle length and $\beta$.
    
  \item The final reduced area increases as the shear rate decreases. In
    the limit of zero shear rate (quiescent flow), the semi-permeable
    vesicles always tends to a circle

  \item The inclination angle of a semi-permeable vesicle in a shear
    flow is greater than the inclination angle of a clean vesicle with
    the same reduced area as the final reduced area of the
    semi-permeable vesicle.

  \item We are able to look at the flux across the interface and
    characterize where there is inflow and outflow.

  \item For $\beta$ that depends on arclength, we have observed both a
    breathing motion (semi-permeable only on opposite sides), and
    migration (semi-permeable only on about 25\% of the vesicle).
    
  \item Compare area results to analytic equation?
\end{itemize}

\subsection{Content permeability rate}


\subsection{Variable permeability rate}

%%%%%%%%%%%%%%%%%%%%%%%%%%%%%%%%%%%%%%%%%%%%%%%%%%%%%%%%%%%%%%%%%%%%%%%%
\section{Conclusions}

%%%%%%%%%%%%%%%%%%%%%%%%%%%%%%%%%%%%%%%%%%%%%%%%%%%%%%%%%%%%%%%%%%%%%%%%
\section*{Conflicts of interest}
In accordance with our policy on
\href{http://www.rsc.org/journals-books-databases/journal-authors-reviewers/author-responsibilities/#code-of-conduct}{Conflicts
of interest} please ensure that a conflicts of interest statement is
included in your manuscript here.  Please note that this statement is
required for all submitted manuscripts.  If no conflicts exist, please
state that ``There are no conflicts to declare''.

%%%%%%%%%%%%%%%%%%%%%%%%%%%%%%%%%%%%%%%%%%%%%%%%%%%%%%%%%%%%%%%%%%%%%%%%
\section*{Acknowledgements}
The Acknowledgements come at the end of an article after Conflicts of interest and before the Notes and references.

%%%END OF MAIN TEXT%%%

%The \balance command can be used to balance the columns on the final page if desired. It should be placed anywhere within the first column of the last page.

\balance

%%%REFERENCES%%%
\bibliography{refs} %You need to replace "rsc" on this line with the name of your .bib file
\bibliographystyle{rsc} %the RSC's .bst file

\end{document}



% Template stuff
%The main text of the article\cite{Mena2000} should appear here.
%
%\subsection{This is the subsection heading style}
%Section headings can be typeset with and without numbers.\cite{Abernethy2003}
%
%\subsubsection{This is the subsubsection style.~~} These headings should end in a full point.  
%
%\paragraph{This is the next level heading.~~} For this level please use \texttt{\textbackslash paragraph}. These headings should also end in a full point.
%
%\section{Graphics and tables}
%\subsection{Graphics}
%Graphics should be inserted on the page where they are first mentioned (unless they are equations, which appear in the flow of the text).\cite{Cotton1999}
%
%\begin{figure}[h]
%\centering
%%  \includegraphics[height=3cm]{example1}
%  \caption{An example figure caption.}
%  \label{fgr:example}
%\end{figure}
%
%\begin{figure*}
% \centering
%% \includegraphics[height=3cm]{example2}
% \caption{A two-column figure.}
% \label{fgr:example2col}
%\end{figure*}
%
%\subsection{Tables}
%Tables typeset in RSC house style do not include vertical lines. Table footnote symbols are lower-case italic letters and are typeset at the bottom of the table. Table captions do not end in a full point.\cite{Arduengo1992,Eisenstein2005}
%
%
%\begin{table}[h]
%\small
%  \caption{\ An example of a caption to accompany a table}
%  \label{tbl:example}
%  \begin{tabular*}{0.48\textwidth}{@{\extracolsep{\fill}}lll}
%    \hline
%    Header one (units) & Header two & Header three \\
%    \hline
%    1 & 2 & 3 \\
%    4 & 5 & 6 \\
%    7 & 8 & 9 \\
%    10 & 11 & 12 \\
%    \hline
%  \end{tabular*}
%\end{table}
%
%Adding notes to tables can be complicated.  Perhaps the easiest method is to generate these manually.\footnote[4]{Footnotes should appear here. These might include comments relevant to but not central to the matter under discussion, limited experimental and spectral data, and crystallographic data.}
%
%\begin{table*}
%\small
%  \caption{\ An example of a caption to accompany a table \textendash\ table captions do not end in a full point}
%  \label{tbl:example}
%  \begin{tabular*}{\textwidth}{@{\extracolsep{\fill}}lllllll}
%    \hline
%    Header one & Header two & Header three & Header four & Header five & Header six  & Header seven\\
%    \hline
%    1 & 2 & 3 & 4 & 5 & 6  & 7\\
%    8 & 9 & 10 & 11 & 12 & 13 & 14 \\
%    15 & 16 & 17 & 18 & 19 & 20 & 21\\
%    \hline
%  \end{tabular*}
%\end{table*}
%
%\section{Equations}
%
%Equations can be typeset inline \textit{e.g.}\ $ y = mx + c$ or displayed with and without numbers:
%
% \[ A = \pi r^2 \]
%
%\begin{equation}
%  \frac{\gamma}{\epsilon x} r^2 = 2r
%\end{equation}
%
%You can also put lists into the text. You can have bulleted or numbered lists of almost any kind. 
%The \texttt{mhchem} package can also be used so that formulae are easy to input: \texttt{\textbackslash ce\{H2SO4\}} gives \ce{H2SO4}. 
%
%For footnotes in the main text of the article please number the footnotes to avoid duplicate symbols. \textit{e.g.}\ \texttt{\textbackslash footnote[num]\{your text\}}. The corresponding author $\ast$ counts as footnote 1, ESI as footnote 2, \textit{e.g.}\ if there is no ESI, please start at [num]=[2], if ESI is cited in the title please start at [num]=[3] \textit{etc.} Please also cite the ESI within the main body of the text using \dag.
