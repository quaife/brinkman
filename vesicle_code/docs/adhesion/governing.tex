\section{Governing Equations \label{sec:ge}}
We consider suspensions of locally inextensible vesicles in an unbounded
two-dimensional viscous fluid.  For simplicity, we assume the fluid
viscosity both inside and outside the vesicles is constant, but our
method can be easily adjusted to account for a viscosity contrast.
Individual vesicles are denoted as $\gamma_j$, $j=1,\ldots,M$, and they
are parameterized in arclength as $\xx_j(s,t)$.  The union of all
vesicles is denoted by $\gamma$.  Given a background velocity
$\uu_\infty$, the governing equations are
\begin{equation*}
\begin{aligned}
  \mu \Delta \uu = \nabla p, \quad &\xx \in \RR^2,
    &&\mbox{\em conservation of momentum} \\
  \nabla \cdot \uu = 0, \quad &\xx \in \RR^2, 
    &&\mbox{\em conservation of mass} \\
  \uu \rightarrow \uu_\infty, \quad &\|\xx\| \rightarrow \infty,
    &&\mbox{\em far-field condition} \\
  \uu(\xx,t) = \dot{\xx}, \quad &\xx \in \gamma,
    &&\mbox{\em velocity continuity} \\
  \xx_s \cdot \uu_s =0, \quad &\xx \in \gamma,
    &&\mbox{\em local inextensibility} \\
  \llbracket T \rrbracket \nn = \xxi, \quad &\xx \in \gamma,
    &&\mbox{\em non-zero traction jump}
\end{aligned}
\end{equation*}
where $\llbracket T \rrbracket$ is the jump in the stress tensor, and
$\xxi$ is the traction jump that is the sum of the bending, stretching,
and adhesion forces.  The bending force, arising from the Helfrich
energy model, is $\BB \xx = -\kappa_b \xx_{ssss}$, where $\kappa_b$ is
the bending rigidity modulus which we set to be 1 for all examples. The
stretching force is $\TT \sigma = (\sigma \xx_s)_s$, where the tension,
$\sigma$, acts as a Lagrange multiplier to satisfy the local
extensibility constraint.  The resistance to bending and stretching are
standard assumptions of vesicles.  In this work, we include an adhesive
force, $\AA \xx$, that we now describe.

%%%%%%%%%%%%%%%%%%%%%%%%%%%%%%%%%%%%%%%%%%%%%%%%%%%%%%%%%%%%%%%%%%%%%%%%
\subsection{Adhesion Model}
Sukumaran and Seifert~\cite{suk-sei2001} consider adhesion between a
single three-dimensional vesicle and a solid wall at $z=0$.  They use
the adhesion potential of the function form
\begin{align}
\label{eq:adhesion_potential}
  \phi(z) = H \left[ 
    \left(\frac{\delta}{z}\right)^m - \frac{m}{n} \left(\frac{\delta}{z}\right)^n \right],
\end{align}
where $z$ is the distance between the solid wall and the vesicle,
$\mathcal{H}$ is the Hamaker constant, $\delta$ is the adhesion length
scale, and $(m,n) = (4,2)$.  We generalize this model to an unbounded
suspension of vesicles. We assume that the adhesion force applies
between all pairs of points on different vesicles, so we are interested
in Hamaker constants that are appropriate for lipid-lipid interactions.
We use Hamaker constants ranging from $\mathcal{O}(10^{-1})$ to
$\mathcal{O}(10^0)$ times the bending rigidity modulus.  In
Appendix~\ref{sec:appendixA}, the adhesive force at a point $\xx$ on
vesicle $j$ is shown to be
\begin{align}
  \AA\xx:=-\mathcal{H} m \delta^{n}\sum_{\substack{k=1 \\ k \neq j}}^M 
  \int_{\gamma_k} \frac{\xx - \yy}{\|\xx - \yy\|^{m+2}} 
  \left(\delta^{m-n} - \|\xx - \yy\|^{m-n} \right) ds_\yy.
  \label{eqn:adhesionForce}
\end{align}
The adhesion model is illustrated in Figure~\ref{fig:adhesionModel}.

\begin{figure}[htp]
  \begin{minipage}{0.31\textwidth}
    \centering
    \includegraphics[height=5cm,trim={0cm 0cm 2cm 0cm},clip]{figs/adhesionPotential.pdf}
  \end{minipage}
  \hfill
  \begin{minipage}{0.31\textwidth}
    \centering
    \includegraphics[height=5cm]{figs/configCartoon.pdf}
  \end{minipage}
  \hfill
  \begin{minipage}{0.31\textwidth}
    \centering
    \includegraphics[height=5cm,trim={3cm 0cm 2cm 0cm},clip]{figs/Adhesion_Force.png}
  \end{minipage}
  \caption{\label{fig:adhesionModel} {\em Left}: The adhesion potential
  $\phi(z)$ in equation~\eqref{eq:adhesion_potential}.  Distances less
  than $\delta$ are repulsive and distances greater than $\delta$ are
  attractive.  {\em Center}: Each point on a vesicle is repelled or
  attracted by all points on the other vesicles.  {\em Right}: The
  resulting adhesive force is found by integrating over all points on
  all other vesicles (equation~\eqref{eqn:adhesionForce}).  The color is
  the dot product of the adhesion force with the vector joining the
  vesicles.  Therefore, the green and yellow regions are repelled by the
  other vesicle while the blue regions are attracted to the other
  vesicle.}
\end{figure}

The exponents $(m,n)$ depend on the geometry of the two objects under
adhesion~\cite{Book_IntermolecularSurfaceForces}: $(m,n)=(4,2)$
corresponds to two flat, planar surfaces interacting with each under a
long-range attraction and a short-range repulsion. On the other hand,
$(m,n) = (12,6)$ corresponds to the L.-J.~potential between two molecules.
In our case, the adhesion potential in the integral of
equation~\eqref{eqn:adhesionForce} is between two small patches of lipid
bilayer membrane (because the lipid molecules are coarse-grained in the
continuum modeling).  Thus a reasonable choice for $(m,n)$ between two
coarse-grained membrane patches would be between those for two planar
surfaces and two point molecules. 
 
Large value of $m$ corresponds to a sharp increase in the repulsion
force as two objects are within the separation distance $\delta$.  This
poses a numerical challenge since the problem becomes stiff (i.e.,
requires a very small time step) for large $m$.  The adaptive
time-stepping BIE scheme makes it possible to simulate vesicle adhesive
dynamics with specified numerical precision for reasonable computation
time.  We explore several combinations of $(m,n)$ in the simulations of
two vesicles forming a doublet in a quiescent flow.  We found very
little difference in both the dynamic evolution and equilibrium
configuration between $(m,n)=(8,6)$ and $(m,n)=(4,2)$.  Therefore, in
this work, we use $(m,n) = (4,2)$ to regulate the stiffness introduced
by the adhesive force.

%%%%%%%%%%%%%%%%%%%%%%%%%%%%%%%%%%%%%%%%%%%%%%%%%%%%%%%%%%%%%%%%%%%%%%%%
\subsection{Integral Equation Formulation}
Because the fluid equations are linear and elliptic, a boundary integral
equation formulation is possible.  This has several advantages,
including that only the vesicle interfaces require discretization.  We
start by defining the Stokes single-layer potential
\begin{align}
  \SS[\ff](\xx) &= \frac{1}{4\pi\mu} \int_\gamma \left(
    -\log \rho + \frac{\rr \otimes \rr}{\rho^2} \right) 
    \ff(\yy) ds_\yy, \quad \xx \in \RR^2,
  \label{eqn:SLP}
\end{align}
where $\rr = \xx - \yy$ and $\rho = \|\rr\|$.  Then, given a vesicle
with a traction jump $\xxi$, the fluid velocity is~\cite{poz1992}
\begin{align*}
  \uu(\xx) = \uu_{\infty}(\xx) + \SS[\xxi](\xx), \quad \xx \in \RR^2.
\end{align*}
The traction jump is
\begin{align*}
  \xxi = -\kappa_b \xx_{ssss} + (\sigma \xx_s)_s + \AA \xx,
\end{align*}
where the three terms correspond to the forces due to bending, tension,
and adhesion, respectively.  Therefore, applying the no-slip boundary
condition on each vesicle, the governing equations are
\begin{align*}
  &\dot{\xx} = \uu_{\infty}(\xx) + \SS[\xxi](\xx), \\
  &\xx_s \cdot \uu_s = 0, \\
  &\xxi = -\kappa_b \xx_{ssss} + (\sigma \xx_s)_s + \AA\xx.
\end{align*}
Defining the bending operator as $\BB[\xx](\ff) = -\kappa_b \ff_{ssss}$,
and the tension operator $\TT[\xx](\sigma) = (\sigma \xx_s)_s$, the
no-slip boundary condition gives
\begin{align*}
  \pderiv{\xx}{t} = \SS \BB \xx + \SS \TT \sigma + \SS \AA \xx.
\end{align*}
We apply the IMEX-Euler method 
\begin{align*}
  \frac{\xx^{N+1} - \xx^N}{\Delta t} = \SS^N \BB^N \xx^{N+1} + 
  \SS^N \TT^N \sigma^{N+1} + \SS^N \AA^N \xx^N,
\end{align*}
that discretizes both the bending and tension terms
semi-implicitly~\cite{qua-bir2014}, and the adhesive term explicitly.
Therefore, a single time step requires solving
\begin{align*}
  \xx^{N+1} - \Delta t \SS^N \BB^N \xx^{N+1} - 
    \Delta t \SS^N \TT^N \sigma^{N+1} = \xx^N + 
    \Delta t \SS^N \AA^N \xx^N,
\end{align*}
along with the inextensibility constraint that is discretized as
\begin{align*}
  \xx_s^{N} \cdot \xx_{s}^{N+1} = 1.
\end{align*}
We discretize the vesicles at a set of collocation points, compute the
bending and tension terms with Fourier differentiation, and apply Alpert
quadrature~\cite{alp1999} to the weakly-singular single-layer potential
$\SS$.  The source and target points of the adhesion force never
coincide since they are always on different vesicles, so the adhesion
force~\eqref{eqn:adhesionForce} is computed with the spectrally accurate
trapezoid rule~\cite{tre-wei2014}.  

The dynamics of a doublet undergoes many different time scales over time
horizons that are sufficiently large to characterize the formation of a
doublet and its rheological properties.  Therefore, time adaptivity is
crucial so that a user-specified tolerance is achieved without using a
guess-and-check procedure to find an appropriately small fixed time step
size.  To control the error and achieve second-order accuracy in time,
we apply a time adaptive spectral deferred correction
method~\cite{quaife2016adaptive}. 

%%%%%%%%%%%%%%%%%%%%%%%%%%%%%%%%%%%%%%%%%%%%%%%%%%%%%%%%%%%%%%%%%%%%%%%%
\subsection{The Shear and Normal Stresses}
To characterize the rheological properties of the suspension of a
doublet, it is necessary to compute the shear and normal stresses.  At a
point $\xx \in \RR^2 \backslash \gamma$ in the fluid bulk, the velocity
$\uu(\xx)$, pressure $p(\xx)$, and total stress tensor $T = - p I +
\nabla \uu + \nabla \uu^T$ of the single-layer potential~\eqref{eqn:SLP}
are
\begin{align*}
  \uu(\xx) &= \SS_\uu[\ff](\xx) = \frac{1}{4\pi\mu}\int_{\gamma} \left( 
    -\log \rho + \frac{\rr \otimes \rr}{\rho^2} \right) 
    \ff ds_\yy, \\
    p(\xx) &= \SS_p[\ff](\xx) = \frac{1}{2\pi} \int_{\gamma} 
    \frac{\rr \cdot \ff}{\rho^2} ds_\yy, \\
    T(\xx) &= \SS_T[\ff](\xx) = -\frac{1}{\pi}\int_{\gamma}
      \frac{\rr \cdot \ff}{\rho^2} 
      \frac{\rr \otimes \rr}{\rho^2} ds_\yy.
\end{align*}
To compute the stress tensor on a vesicle, there is a jump in the layer
potential $\SS_T$, and the stress tensor is
\begin{align*}
  T(\xx) = \frac{1}{2}(\nn \otimes \ff) + \frac{1}{2} \left(
    \tt \otimes \left[
    \begin{array}{cc}
      2 t_x t_y & t_y^2 - t_x^2 \\
      t_y^2 - t_x^2 & -2 t_x t_y
    \end{array}
    \right] \ff \right) + 
  \SS_T[\ff](\xx), \quad \xx \in \gamma,
\end{align*}
where $\nn$ and $\tt$ are the vesicle's outward unit normal and tangent
vector, respectively.  With the stress tensor computed for $\xx \in
\gamma$, the shear and normal stresses are
\begin{align*}
  \tau^\tt = (T \nn) \cdot \tt, \qquad
  \tau^\nn = (T \nn) \cdot \nn.
\end{align*}

