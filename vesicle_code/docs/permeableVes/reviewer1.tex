\documentclass[11pt]{article}

\usepackage{fullpage}
\usepackage{color}
\newcommand{\response}[1]{\textcolor{blue}{#1}}
\newcommand{\note}[1]{\textcolor{red}{#1}}

\begin{document}
We thank the reviewer for reviewing our manuscript \textit{Hydrodynamics
of a semipermeable inextensible membrane under flow and confinement}.
Below is a point-by-point response addressing each of the referee's
comments.
\begin{enumerate}
  \item It is stated in the introduction that stretching (due to
    poration $\ldots$) give rise to enhanced membrane permeability to
    both water and macromolecules such as ATP. Regarding ATP this is
    possible. In normal conditions ATP is released thanks to protein
    change of conformation that open channels and I am not sure the
    evoked permeability is a similar phenomenon. A few words of
    clarification are needed. \\ \\
    \response{A response}

  \item What cause more permeation? (i) hydrodynamic stress of
    suspending fluid, (ii) tension (normal component of force, say
    Lagrange multiplier), (ii) or the tangential force component due to
    tension field? It would be nice to check this point. \\ \\
    \response{A response}

  \item Figure 5 is not well explained. Depending on Capillary number it
    seems that for all reduced area converge to the same (solid line)
    value or continue to decrease (dashed line). I suggest both in text
    and in caption to discuss this issue more clearly. The result
    (Fig.~5a) is also not intuitive: when capillary number is large (10)
    there is a final reduced area, whereas for smaller value 1, the
    final reduced area is smaller. I would have expected that smaller
    $Ca_E$ would empty less the vesicle. Is there any interpretation? \\
    \\
    \response{In the quiescent example, we prove and demonstrate
    numerically that the steady state vesicle reduced area is
    independent of the initial reduced area. Figure 5a shows that this
    result holds in the presence of flow. It also shows that the
    steady state reduced area depends on the capillary number. However,
    the entries in the legend of Figure 5a were accidentally reversed in
    the original submission---this has been corrected. Therefore, the
    reviewers intuition is correct that smaller capillary number results
    in less emptying of the vesicle.}

  \item In Fig.~7b reference is made to steady state? I guess there is
    still water flow across membrane? Or the total flux is zero? In
    other word is the volume of the shapes shown there the same along
    the blue solid line? Then I am confused with inset c which shows
    variation of alpha? My question: are shapes in b along blue line
    long term ones? Is alpha constant there? Please clarify. \\ \\
    \response{A response}
    \note{Is the 2D Green's function the same as the 3D green's function
    in cylindrical coordinates.}

  \item I do not quite understand why permeation time scale is much
    lower for confined situations? Is pressure drop fixed, as compared
    to previous cases? Does cell see stronger stress now? Why? If the
    driving force is the same I do not expect such dramatic change of
    time scale, since the mean shear stress experienced by the cell
    remains of the same order, as before. A clear discussion of origin
    (from stress values point of view, and why confinement would enhance
    imposed stress) is necessary. This section is poorly discussed. \\
    \\
    \response{A response}

\end{enumerate}


\end{document}
