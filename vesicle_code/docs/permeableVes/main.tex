\documentclass[aps,prl,showpacs]{revtex4}
\usepackage{amsmath,amsfonts,bm, color}
\usepackage{graphicx,epsfig}
%\input vs_macros.tex
%Dfig puts a figure of specified size at the place indicated.
\newcommand{\Dfig}[2]{\epsfig{figure=#1,width=#2cm}}

 
%\Eonefigs puts one figure at the top of a page. 3 arguments.
%Usage:
%\Eonefigs{file fig}{caption}{size in inches}
%the file must be eps, but names must be given 
%WITHOUT .eps extension

\newenvironment{Eonefigs}[3]
{        \begin{figure} [t]
          \begin{center}
              \includegraphics[width=0.5\textwidth]{#3} \\
          \vspace{-0.21in}
          \caption{{\footnotesize #2}}
          \label{#1}
          \end{center}
        \vspace{-.3in}
        \end{figure}
}{}




\newcommand{\co}[1]{\marginpar{\scriptsize \textcolor{red}{#1}}} % red comment

\input my_macros2.tex	

\begin{document}

\title{Semi-permeable vesicles: mathematical formulation, numerics and physics}
\author{}

\date{\today}

\begin{abstract}
We describe a mathematical model for a semi-permeable vesicle and derive a boundary integral equation formulation for simulating its hydrodynamics.  
\end{abstract}
\maketitle


\section{Problem formulation} \label{sc:formulate}
Consider a semi-permeable vesicle suspended in an unbounded viscous
fluid domain, subjected to an imposed flow $\mbu_\infty(\mbx)$, for any
$\mbx \in \RR^2$.  Assume that the interior and exterior fluids have the same viscosity $\mu$. Let $\mbx$ be the position of the interface $\gamma$, $\mbu$ the fluid velocity and $p$ the pressure. In the vanishing Reynolds number limit, the governing equations for the ambient fluid can be written as: 
%
\begin{subequations}
\begin{align}
-\nabla p + \mu \triangle \mbu = 0 \quad\text{in}\quad
  \RR^2\setminus\gamma  &  \\
% 
\nabla \cdot \mbu = 0  \quad\text{in}\quad \RR^2\setminus\gamma &\\
%
\mbu(\mbx) \rightarrow \mbu_\infty(\mbx) \quad \text{as} \quad  \norm{\mbx} \rightarrow \infty & 
 \end{align} \label{eqn:governing}
\end{subequations}%
%
The modified jump conditions in the case of a semi-permeable vesicle can be summarized as follows.  Let $\mbn$ the outward normal to $\gamma$ and $\jump{\cdot}$ denote the jump across the interface. The jump condition for the hydrodynamic stress remains the same:
%
%Then, define
%\beq \mbf_i = \lim_{\mbx \rightarrow \gamma^-} \sigma (\mbx)\cdot\mbn,  \eeq
%\beq \mbf_e =  \lim_{\mbx \rightarrow \gamma^+} \sigma (\mbx)\cdot\mbn, \eeq
%\beq \jump{\mbv\cdot \mbt} = 0 \eeq
%\beq \jump{\mbv\cdot\mbn} = \beta (\mbf_i + \mbf_e)\cdot\mbn \eeq
\beq \jump{\Sigma\cdot\mbn} = \mbf_\text{mem}, \eeq
where $\Sigma$ is the hydrodynamic stress tensor. The fluid velocity is continuous across the membrane i.e.,
\beq \jump{\mbu} = 0. \eeq
However, the kinematic condition is modified to account for the fact that the fluid can move through the membrane. A standard choice appears to be
%
\beq \mbu - \dot{\mbx} = - \beta (\mbf_\text{mem} \cdot \mbn) \mbn \quad\text{on}\quad\gamma. \eeq
%
Here, $\beta$ is a constant and the membrane force is the usual tension + bending. 

\section{Integral equation formulation}
There are multiple ways to convert the PDE problem into integral
equations. Based on my derivation, the coupled set of
integro-differential equations satisfy the given interfacial conditions:
\beq 
  \dot{\mbx} = \mbu_\infty(\mbx) + \beta (\mbf_\text{mem}\cdot\mbn)\mbn + \sgl [\mbf_\text{mem}], 
  \label{eqn:vesVelocity}
\eeq
\beq 
  \mbx_s\cdot\dot{\mbx}_s = 0 
\eeq
This satisfies all the jump conditions. As usual, $\sgl$ is the single layer potential. 

\section{Numerics}
To address stiffness associated with high-order derivatives,
equation~\eqref{eqn:vesVelocity} is discretized with a semi-implicit
time stepping method similar to our previous works.  First, we introduce
the notation
\begin{align}
  B[\mbx]\mbf &= -\kappa_b\frac{d^4}{ds^4} \mbf, \\
  T[\mbx]\sigma &= (\sigma \mbx_s)_s, \\
  P[\mbx]\mbf &= (\mbf \cdot \mbn) \mbn, \\
  D[\mbx]\mbf &= \mbx_s \cdot \mbf_s,
\end{align}
where $\mbn$ is the outward unit normal of the vesicle
parameterized by $\mbx$, and $d/ds$ is its arclength derivative.
Dropping the dependence of the operators on $\mbx$, the dynamics of the
vesicle are governed by
\begin{align}
  \dot{\mbx} &= \mbu_{\infty}(\mbx) + 
  \beta P(B\mbx + T\sigma) + \sgl(B\mbx + T\sigma), \\
  D \dot{\mbx} &= 0.
\end{align}
Using the notation $B^N$ to denote the bending operator due to the
vesicle configuration at time $t_N$, and similar notation for the other
operators, we apply the semi-implicit time stepping method
\begin{align}  
  \frac{\mbx^{N+1} - \mbx^N}{\Delta t} = \mbu_\infty(\mbx^N) 
  + \beta P^N(B^N\mbx^{N+1} + T^N\sigma^{N+1}) 
  + \sgl^N(B^N\mbx^{N+1} + T^N\sigma^{N+1}).
\end{align}
Applying $D$ to the first-order time discretization of
equation~\eqref{eqn:vesVelocity}, the inextensibility condition is
\begin{align*}
  &D^N(\mbx^{N+1} - \mbx^N) = 0 \\ 
  \Rightarrow &D^N\mbx^{N+1} = D^N \mbx^N \\ 
  \Rightarrow &D^N\mbx^{N+1} = 1.
\end{align*}
Finally, we write the time stepping method in matrix form as
\begin{align}
  A \left(
    \begin{array}{c}
      \mbx^{N+1} \\ \sigma^{N+1}
    \end{array}
  \right) = 
  \left(
    \begin{array}{c}
      \mbx^{N} + \Delta t \mbu_\infty(\mbx^N) \\ \mathbf{1}
    \end{array}
  \right),
\end{align}
where
\begin{align}
 A = \left(
  \begin{array}{cc}
    I + \Delta t \beta P^N B^N + \Delta t \sgl^N B^N & 
    -\Delta t P^N T^N - \Delta t \sgl^N T^N \\
    D^N & 0
  \end{array}
  \right).
\end{align}

\section{Area of a Vesicle}
The area inside a single vesicle $\gamma$ parameterized with $\mbx$ is
\begin{align}
  A(t) = \int_{\gamma} (\mbx \cdot \mbn)\, ds = 
    \int_{0}^{2\pi} \left(\mbx \cdot \pderiv{\mbx}{\theta}^\perp\right)
    \, d\theta.
\end{align}
Taking the time derivative of $A(t)$,
\begin{align}
  \dot{A}(t) &= \int_{0}^{2\pi} \left(
    \dot{\mbx} \cdot \pderiv{\mbx}{\theta}^{\perp} + 
    \mbx \cdot \pderiv{\dot{\mbx}}{\theta}^{\perp}\right) \\
    \, d\theta.
  &= \int_{0}^{2\pi} \left(
    (\dot{\mbx} \cdot \mbn)\|\mbx'(\theta)\| \right) d\theta +
     \int_{0}^{2\pi}
    \mbx \cdot \frac{d}{dt}\left(\mbn |\frac{d\mbx}{d\theta}| \right)
\end{align}

\section{Including a concentration field}
Based on the work of Yao and Mori~\cite{yao-mor2017}, we introduce a
chemical concentration field $c$.  The concentration field is governed
by the advection-diffusion equation
\begin{align}
  \pderiv{c}{t} + \nabla \cdot (\mbu c) = D\Delta c.
\end{align}
If the concentration is diffusion dominated (zero Peclet number), then
the concentration satisfies
\begin{align}
  \Delta c = 0, \quad \mbx \in \RR^2.
\end{align}
The boundary conditions of the concentration field on the vesicle
$\gamma$ are
\begin{align}
  (\mbu c - D\nabla c) \cdot \mbn = c \pderiv{\mbx}{t} \cdot \mbn + 
    j_c + j_p,
\end{align}
where $j_c$ and $j_p$ are the transmembrane chemical flux due to passive
transport and active transport (pumps), respectively.  In the zero
Peclet number limit, the absence of an active transport mechanism, and
the choice for the passive flux, the boundary condition of the
concentration field is
\begin{align}
  -D\pderiv{c}{\mbn} = c \pderiv{\mbx}{t} + 
    k_c \jump{c}, \quad \mbx \in \gamma.
\end{align}
To couple the vesicle dynamics to the concentration, there is a
difference between the fluid velocity and vesicle velocity
\begin{align}
  \mbu - \pderiv{\mbx}{t} = j_w \mbn,
\end{align}
where $j_w$ is the water flux through the vesicle.  This is the
osmophoretic flow.  In~\cite{yao-mor2017}, this flux is
\begin{align}
  j_w = -k_w(RT\jump{c} + \mbf_\text{mem} \cdot \mbn).
\end{align}
In our current implementation, we drop the $RT\jump{c}$ term, and
let $\beta = k_w$.



\section{Preliminary results}

\begin{figure}
	\centering
	\includegraphics[width=.9\textwidth]{figures/1.jpg}
	\includegraphics[width=.9\textwidth]{figures/2.jpg}
	\includegraphics[width=.9\textwidth]{figures/3.jpg}
	\caption{Reduced area as a function of time for $\beta = [0, 10^{-2},10^{-1}, 1]$ with a shear flow rate $\chi = 0$.}
%	\label{Shear0}
\end{figure}

\begin{figure}
	\centering
	\includegraphics[width=.9\textwidth]{figures/4.jpg}
	\includegraphics[width=.9\textwidth]{figures/5.jpg}
	\includegraphics[width=.9\textwidth]{figures/6.jpg}
	\caption{Reduced area as a function of time for $\beta = [0, 10^{-2},10^{-1}, 1]$ with a shear flow rate $\chi = 0.5$.}
%	\label{Shear0}
\end{figure}

\begin{figure}
	\centering
	\includegraphics[width=.9\textwidth]{figures/7.jpg}
	\includegraphics[width=.9\textwidth]{figures/8.jpg}
	\includegraphics[width=.9\textwidth]{figures/9.jpg}
	\caption{Reduced area as a function of time for $\beta = [0, 10^{-2},10^{-1}, 1]$ with a shear flow rate $\chi = 1$.}
%	\label{Shear0}
\end{figure}

\begin{figure}
	\centering
	\includegraphics[width=.9\textwidth]{figures/10.jpg}
	\includegraphics[width=.9\textwidth]{figures/11.jpg}
	\includegraphics[width=.9\textwidth]{figures/12.jpg}
	\caption{Reduced area as a function of time for $\beta = [0, 10^{-2},10^{-1}, 1]$ with a shear flow rate $\chi = 2$.}
%	\label{Shear0}
\end{figure}
 
 
 \begin{figure}
 	\centering
 	\includegraphics[width=.9\textwidth]{figures/BE1.jpg}
 	\includegraphics[width=.9\textwidth]{figures/BE2.jpg}
 	\includegraphics[width=.9\textwidth]{figures/BE3.jpg}
 	\caption{Bending energy as a function of time for $\beta = [0, 10^{-2},10^{-1}, 1]$ with a shear flow rate $\chi = 0$.}
% 	\label{Shear0}
 \end{figure}
 
 \begin{figure}
 	\centering
 	\includegraphics[width=.9\textwidth]{figures/BE4.jpg}
 	\includegraphics[width=.9\textwidth]{figures/BE5.jpg}
 	\includegraphics[width=.9\textwidth]{figures/BE6.jpg}
 	\caption{Bending energy as a function of time for $\beta = [0, 10^{-2},10^{-1}, 1]$ with a shear flow rate $\chi = 0.5$.}
% 	\label{Shear0}
 \end{figure}
 
 \begin{figure}
 	\centering
 	\includegraphics[width=.9\textwidth]{figures/BE7.jpg}
 	\includegraphics[width=.9\textwidth]{figures/BE8.jpg}
 	\includegraphics[width=.9\textwidth]{figures/BE9.jpg}
 	\caption{Bending energy as a function of time for $\beta = [0, 10^{-2},10^{-1}, 1]$ with a shear flow rate $\chi = 1$.}
% 	\label{Shear0}
 \end{figure}
 
 \begin{figure}
 	\centering
 	\includegraphics[width=.9\textwidth]{figures/BE10.jpg}
 	\includegraphics[width=.9\textwidth]{figures/BE11.jpg}
 	\includegraphics[width=.9\textwidth]{figures/BE12.jpg}
 	\caption{Bending energy as a function of time for $\beta = [0, 10^{-2},10^{-1}, 1]$ with a shear flow rate $\chi = 2$.}
 	\label{Shear0}
 \end{figure}

 \bibliographystyle{plain}
 \bibliography{refs}


\end{document}
