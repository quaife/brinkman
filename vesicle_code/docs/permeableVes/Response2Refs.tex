\documentclass[12pt]{article}
\usepackage{amsmath,amsfonts,bm,wrapfig}
\usepackage[left=1in,right=1in,top=1in,nohead]{geometry}
\usepackage[dvips]{graphicx}
\usepackage[
pdftitle={nsf si2 bie},
pdfcreator={pdftex},
pdfsubject={paper},
pdfkeywords={some,topics},
hyperindex = {true},  % turn to false for final
colorlinks = {true},  % turn to false for final
linkcolor = {blue},
pagecolor = {blue},
citecolor = {blue}
]{hyperref}
\usepackage{nsf}
\usepackage{paralist}

\usepackage[T1]{fontenc}
\usepackage{ae,aecompl}

\begin{document}

{\large \bf Response to the review report for PRF\_FB10127 Quaife: Hydrodynamics of a semipermeable inextensible membrane under flow and confinement}
\vspace{0.5cm}

We have revised the manuscript ``Hydrodynamics of a semipermeable inextensible membrane under flow and confinement"  
by Bryan Quaife, Ashley Gannon and Y.-N. Young, 
based on 
the referees' very helpful feedback.  
Thanks to the referees' comments
and suggestions, we believe the revised manuscript is much
improved. 
In the revised manuscript we ensure that
all of their concerns and questions are properly addressed.
Below is a summary of the revisions made in the new manuscript.
The detailed responses to comments of referees are listed in 
Response to Referee 1 and Referee 2.
We hope that the referees and the
editor will now find the paper suitable for publication in 
Physical Review Fluids.

\vspace{0.5cm}
{\large \bf List of changes}
In the following we refer to 
Referee 1 who has four comments and Referee 2 who has a list of comments. In the revised manuscript the modifications in the text are in blue.

\begin{enumerate}
\item{change 1}
\item{change 2}
\end{enumerate}

\newpage
\vspace{0.5cm}
{\large \bf Response to Referee 1:}
\vspace{0.5cm}

We thank the referee for his/her assessment of our work.  
The following is
our response to each of the specific comments and suggestions.


{\bf The referee wrote:
The paper deals with effect of membrane permeation under equilibrium
and flow conditions. This question is rarely addressed in literature.
The authors highlight the fact that red blood cells have a long life
time, and permeation may become decisive. The work report on several
interesting issues and should be of interest to the community working
on cell dynamics under flow. The paper is clearly written and results
very sound. I support publication after the authors have taken into
account the following points.
}

\noindent
Response: We thank the referee for the confirmation that the topic is of interest to researchers working on cell dynamics under flow.

\begin{enumerate}
\item {\bf 
The referee wrote:
``It is stated in the introduction that stretching (due to poration…)
give rise to enhanced membrane permeability to both water and
macromolecules such as ATP. Regarding ATP this is possible. In normal
conditions ATP is released thanks to protein change of conformation
that open channels and I am not sure the evoked permeability is a
similar phenomenon. A few words of clarification are needed."}

\noindent
Response: In recent experiments and simulations by Yaling Liu (Lehigh University), it is reported that under very large stress the spectrin network under a red blood cell membrane can be destroyed and the red blood cell membrane can rupture. As a result the hemoglobin inside the RBC is found to reduce in this process. We now see that the source of confusion arises from the fact that we associate such leakage to enhanced membrane permeability to macromolecules. 

We agree with the referee that the ATP release is mostly understood when it is through the coupling between Px1 (activated by membrane tension) and CFTR (activated by membrane deformation), and it is not well understood yet how ATP can be released by diffusing through the membrane. We have thus modified this sentence to avoid confusion.

\noindent
Location: 

\item {\bf
The referee wrote:
``What cause more permeation? (i) hydrodynamic stress of suspending
fluid, (ii) tension (normal component of force, say Lagrange
multiplier), (ii) or the tangential force component due to tension
field? It would be nice to check this point."}

\noindent
Response: We thank the referee for this point. In our formulation, the permeation flux is proportional to the hydrostatic pressure and the normal component of the membrane force, which is the capillary pressure due to the Lagrange multiplier tension. The tangential force component due to tension field does not contribute to the permeation flux. To elucidate the relative contribution between the hydrostatic pressure and the normal component of the membrane force, we have added a short paragraph to illustrate how the relative contributions are in all cases.

\noindent
Location:


\item {\bf 
The referee wrote:
``Figure 5 is not well explained. Depending on Capillary number it
seems that for all reduced area converge to the same (solid line)
value or continue to decrease (dashed line). I suggest both in text
and in caption to discuss this issue more clearly. Th result (Fig. 5a)
is also not intuitive: when capillary number is large (10) there is a
final reduced area, whereas for smaller value 1, the final reduced
area is smaller. I would have expected that smaller Cae would empty
less the vesicle. Is there any interpretation?"}

\noindent
Response: We have mislabeled the $Ca_E$ in figure 5:  $Ca_E=1$ for the solid curves and the $Ca_E=10$ for the dashed curves. To address the effect of Capillary number on the relaxation of a semipermeable vesicle, we have continued the simulations for the cases of $Ca_E = 10$ much further to better illustrate the final reduced area at equilibrium.

\item{\bf
The referee wrote:
``In Fig. 7 B reference is made to steady state? I guess there is
still water flow across membrane? Or the total flux is zero? In other
word is the volume of the shapes shown there the same along the blue
solid line? Then I am confused with inset c which shows variation of
alpha? My question: are shapes in b along blue line long term ones? Is
alpha constant there? Please clarify."}

\noindent
Response: The shapes along the blue curve in Fig. 7B are equilibrium shapes that enclose a constant area but with an efflux (blue) and an influx (red) that adds to zero. Inset c illustrates the evolution from initial  the configuration (I) to the equilibrium (shape III), which corresponds to the case with the largest displacement on the blue curve. To better present all these results, we have taken out the insets (c) and (d) and moved them to the new Fig. 7 A and B. The new Fig. 7 C and D are the old Fig.  7 A and B.

\noindent
Location:

\item{\bf
The referee wrote:
``I do not quite understand why permeation time scale is much lower
for confined situations? Is pressure drop fixed, as compared to
previous cases? Does cell see stronger stress now? Why? If the driving
force is the same I do not expect such dramatic change of time scale,
since the mean shear stress experienced by the cell remains of the
same order, as before. A clear discussion of origin (from stress
values point of view, and why confinement would enhance imposed
stress) is necessary. This section is poorly discussed."}

\noindent
Response:  For the simulations in \S V for the two confinement cases, we have to use higher values of $\beta$ to keep the membrane tension below the threshold value (lysis tension) for membrane poration. {\bf In the manuscript I could not find the values of $\beta$ in \S V.} In addition, the far-field velocity is fixed in these simulations with confinement. We did not fix the pressure jump across the channel in these simulations. We have modified the text to provide a better explanation of this section.

\noindent
Location:

\end{enumerate}

\newpage
\vspace{0.5cm}
{\large \bf Response to Referee 2:}
\vspace{0.5cm}

We thank the referee for his/her assessment of our work.  
The following is
our response to each of the specific comments and suggestions.

{\bf The referee wrote:
In this theoretical work, a 2D vesicle is simulated (via boundary elements) in unbounded and confined flows, while accounting for the finite permeability of the membrane to water. While this additional physics has, expectedly, little to no impact on the long-time physics (shape, tension) of vesicles in unbounded flows, in confined flows a marked dependence is observed on the permeability rate. In straight channel flow, the rate of water loss is proportional to the residence time in the channel, resulting in a reduction of the 2D reduced area at the channel outlet. In contraction flow, a high permeability rate (as well as a large pressure gradient) is required to force the vesicle through the nozzle and emerge in the open chamber downstream.

Overall, the paper is well written and the main results are clearly presented. A decent amount of time is spent covering vesicles in quiescent fluids and unbounded flows, which are less interesting than Section V, which includes the main interesting results on confined flows. However, these earlier sections help to paint a cohesive picture to put these later results in a clear context and, for that reason, their addition is welcome. My only real critique of the article is that all of the results are based on 2D simulations and, therefore, their applicability to real 3D systems is limited. However, the novelty of adding finite membrane permeability to the 2D boundary element context (which, to the best of my knowledge, has not been done before) merits publication of this article in Physical Review Fluids. I am therefore recommending this article for publication subject to optional revision (see my comments below).}


\noindent
Response: We thank the referee for the confirmation that the topic is interesting and results are useful.

\begin{enumerate}

\item {\bf The referee wrote: ``1. I am slightly confused as to how the length and time scales are derived from the parameters discussed on page 4. As the simulation is 2D, it would seem to me that all physical parameters can only be defined on a “per unit of length” basis. E.g., for a 2D vesicle contour, the surface tension is actually a line tension with units of N, not N/m. Could the authors clarify this?
In particular, reference is often made to how long things will take in the results (seconds,
minutes, days, weeks, etc.), which is why I ask how these time scales are obtained."}

\noindent
Response: 

\noindent
Locations: 


\item {\bf The referee wrote: ``2. Could the authors shed some insight as to how these 2D simulations could be reliably connected to real experiments on any quantitative basis? Or is the connection purely
qualitative?"}

\noindent
Response: 

\noindent
Location: Second paragraph in the new Introduction in page 2.

\item {\bf The referee wrote: ``On the last page: “We observe lose of”è“We observe loss of”."}

\noindent
Response: \end{enumerate}



\end{document}
