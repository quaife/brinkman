\documentclass[aps,prl,twocolumn,showpacs]{revtex4}
\usepackage{amsmath,amsfonts,bm, color}
\usepackage{graphicx,epsfig}
%\input vs_macros.tex
%Dfig puts a figure of specified size at the place indicated.
\newcommand{\Dfig}[2]{\epsfig{figure=#1,width=#2cm}}

 
%\Eonefigs puts one figure at the top of a page. 3 arguments.
%Usage:
%\Eonefigs{file fig}{caption}{size in inches}
%the file must be eps, but names must be given 
%WITHOUT .eps extension

\newenvironment{Eonefigs}[3]
{        \begin{figure} [t]
          \begin{center}
              \includegraphics[width=0.5\textwidth]{#3} \\
          \vspace{-0.21in}
          \caption{{\footnotesize #2}}
          \label{#1}
          \end{center}
        \vspace{-.3in}
        \end{figure}
}{}




\newcommand{\co}[1]{\marginpar{\scriptsize \textcolor{red}{#1}}} % red comment

\input my_macros2.tex	

\begin{document}

\title{Semi-permeable vesicles: mathematical formulation, numerics and physics}
\author{}

\date{\today}

\begin{abstract}
We describe a mathematical model for a semi-permeable vesicle and derive a boundary integral equation formulation for simulating its hydrodynamics.  
\end{abstract}
\maketitle


\section{Problem formulation} \label{sc:formulate}
Consider a semi-permeable vesicle suspended in an unbounded viscous fluid domain, subjected to an imposed flow $\mbu_\infty(\mbx)$, for any $\mbx \in \mathbb{R}^2$.  Assume that the interior and exterior fluids have the same viscosity $\mu$. Let $\mbx$ be the position of the interface $\gamma$, $\mbu$ the fluid velocity and $p$ the pressure. In the vanishing Reynolds number limit, the governing equations for the ambient fluid can be written as: 
%
\begin{subequations}
\begin{align}
-\nabla p + \mu \triangle \mbu = 0 \quad\text{in}\quad \mathbb{R}^2\setminus\gamma  &  \\
% 
\nabla \cdot \mbu = 0  \quad\text{in}\quad \mathbb{R}^2\setminus\gamma &\\
%
\mbu(\mbx) \rightarrow \mbu_\infty(\mbx) \quad \text{as} \quad  \norm{\mbx} \rightarrow \infty & 
 \end{align} \label{eqn:governing}
\end{subequations}%
%
The modified jump conditions in the case of a semi-permeable vesicle can be summarized as follows.  Let $\mbn$ the outward normal to $\gamma$ and $\jump{\cdot}$ denote the jump across the interface. The jump condition for the hydrodynamic stress remains the same:
%
%Then, define
%\beq \mbf_i = \lim_{\mbx \rightarrow \gamma^-} \sigma (\mbx)\cdot\mbn,  \eeq
%\beq \mbf_e =  \lim_{\mbx \rightarrow \gamma^+} \sigma (\mbx)\cdot\mbn, \eeq
%\beq \jump{\mbv\cdot \mbt} = 0 \eeq
%\beq \jump{\mbv\cdot\mbn} = \beta (\mbf_i + \mbf_e)\cdot\mbn \eeq
\beq \jump{\Sigma\cdot\mbn} = \mbf_\text{mem}, \eeq
where $\Sigma$ is the hydrodynamic stress tensor. The fluid velocity is continuous across the membrane i.e.,
\beq \jump{\mbu} = 0. \eeq
However, the kinematic condition is modified to account for the fact that the fluid can move through the membrane. A standard choice appears to be
%
\beq \mbu - \dot{\mbx} = - \beta (\mbf_\text{mem} \cdot \mbn) \mbn \quad\text{on}\quad\gamma. \eeq
%
Here, $\beta$ is a constant and the membrane force is the usual tension + bending. 

\section{Integral equation formulation}
There are multiple ways to convert the PDE problem into integral
equations. Based on my derivation, the coupled set of
integro-differential equations satisfy the given interfacial conditions:
\beq 
  \dot{\mbx} = \mbu_\infty(\mbx) + \beta (\mbf_\text{mem}\cdot\mbn)\mbn + \sgl [\mbf_\text{mem}], 
  \label{eqn:vesVelocity}
\eeq
\beq 
  \mbx_s\cdot\dot{\mbx}_s = 0 
\eeq
This satisfies all the jump conditions. As usual, $\sgl$ is the single layer potential. 

\section{Numerics}
To address stiffness associated with high-order derivatives,
equation~\eqref{eqn:vesVelocity} is discretized with a semi-implicit
time stepping method similar to our previous works.  First, we introduce
the notation
\begin{align}
  B[\mbx]\mbf &= -\kappa_b\frac{d^4}{ds^4} \mbf, \\
  T[\mbx]\sigma &= (\sigma \mbx_s)_s, \\
  P[\mbx]\mbf &= (\mbf \cdot \mbn) \mbn, \\
  D[\mbx]\mbf &= \mbx_s \cdot \mbf_s
\end{align}
where $\mbn$ is the is the outward unit normal of the vesicle
parameterized by $\mbx$, and $d/ds$ is its arclength derivative.
Dropping the dependence of the operators on $\mbx$, the dynamics of the
vesicle are governed by
\begin{align}
  \dot{\mbx} &= \mbu_{\infty}(\mbx) + 
  \beta P(B\mbx + T\sigma) + \sgl(B\mbx + T\sigma), \\
  D \dot{\mbx} &= 0.
\end{align}
Using the notation $B^N$ to denote the bending operator due to the
vesicle configuration at time $t_N$, and similar notation for the other
operators, we apply the semi-implicit time stepping method
\begin{multline}  
  \frac{\mbx^{N+1} - \mbx^N}{\Delta t} = \mbu_\infty(\mbx^N) 
  + \beta P^N(B^N\mbx^{N+1} + T^N\sigma^{N+1})  \\
  + \sgl^N(B^N\mbx^{N+1} + T^N\sigma^{N+1}).
\end{multline}
Applying $D$ to the first-order time discretization of
equation~\eqref{eqn:vesVelocity}, the inextensibility condition is
\begin{align*}
  &D^N(\mbx^{N+1} - \mbx^N) = 0 \\ 
  \Rightarrow &D^N\mbx^{N+1} = D^N \mbx^N \\ 
  \Rightarrow &D^N\mbx^{N+1} = 1.
\end{align*}
Finally, we write the time stepping method in matrix form as
\begin{align}
  A \left(
    \begin{array}{c}
      \mbx^{N+1} \\ \sigma^{N+1}
    \end{array}
  \right) = 
  \left(
    \begin{array}{c}
      \mbx^{N} + \Delta t \mbu_\infty(\mbx^N) \\ \mathbf{1}
    \end{array}
  \right),
\end{align}
where
\beq
 A = \left(
  \begin{array}{cc}
    I + \Delta t \beta P^N B^N + \Delta t \sgl^N B^N & 
    -\Delta t P^N T^N - \Delta t \sgl^N T^N \\
    D^N & 0
  \end{array}
  \right).
\eeq

 
%We obtain the matrix
%\begin{multline}
% \begin{pmatrix}
% 	\mbx^N \\ \textbf{1} 
% \end{pmatrix} = \\
%\begin{pmatrix}
%	I - \Delta t \sgl^N\mathcal{B}^N - \Delta t \beta \mathcal{P}^N\mathcal{B}^N & -\Delta t \sgl^N \mathcal{T}^N-\Delta t\beta \mathcal{P}^N\mathcal{T}^N \\ & \textbf{0}
%\end{pmatrix}
%\begin{pmatrix}
%	\mbx^{N+1} \\ \sigma^{N+1}
%\end{pmatrix} 
%\end{multline}

\end{document}
